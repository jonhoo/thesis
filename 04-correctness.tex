Partial state significantly changes how the underlying dataflow system computes
query results, and without care, might cause the system to violate eventual
consistency. This chapter gives an informal correctness argument for how Noria
preserves eventual consistency by maintaining necessary system invariants for
dataflow of increasing complexity. Section~\ref{s:invariants} introduces system
invariants that Noria must uphold to ensure eventual consistency.
Section~\ref{s:partial:linear} makes the argument for why a single strand of
dataflow yields correct results. Section~\ref{s:partial:diverging} expands that
argument to include dataflow with multiple diverging branches. And
Section~\ref{s:partial:merging} completes the argument by also considering
dataflow where multiple strands join together. Finally,
Section~\ref{s:challenge:sharding} discusses how partial state interacts with
sharded dataflow.

\section{System Invariants}
\label{s:invariants}

The biggest change that partial state introduces is that multiple updates may
now be collapsed and re-processed through the dataflow as a single, consolidated
update in response to an upquery. These represent snapshots of upstream state,
and Noria must ensure that these snapshots do not cause updates to be duplicated
or dropped so that views are eventually consistent. While in theory Noria could
later undo updates it duplicates, or restore updates it drops, Noria has no
mechanism for doing so, and instead aims to not introduce errors that must be
corrected later in the first place.

Upquery responses must represent the current state of an operator at a
particular point in time, and must include all earlier updates, and no later
updates. If an upquery response does not include an earlier update, that update
would be lost, and if it includes a later update, the application of that later
update would cause the update's effects to be duplicated. To that end
Noria upholds the following two safety invariants:

\begin{invariant}
  \label{i:no-spurious}
  All reads reflect each base table change at most once.
\end{invariant}

This invariant ensures that Noria does not accidentally duplicate base table
changes, such as by double-counting an insert or deletion. If this invariant
were violated, a base table insert might be performed twice, leading a view to
perpetually duplicate that row in its result set. Note, however, that is does
not preclude a \emph{value} present in a given base table row from appearing
multiple times in a downstream view. If a query explicitly duplicates
rows with a \texttt{UNION} or a column value appears in multiple output rows
through a \texttt{JOIN}, then each base table \emph{change} is still reflected
at most once.

% For a concrete example of this, consider a query that selects all columns from
% all rows in a given table twice, and takes the union of them. The expected
% output is that each \emph{row} in that table appears twice. However, each
% \emph{change} to the base table\,---\,the inserts that added those
% rows\,---\,is reflected only once. If a given \emph{insert} was reflected more
% than once, then the inserted row would appear more than twice by virtue of
% there (incorrectly) being multiple copies of it in the table.

\begin{invariant}
  \label{i:no-holes}
  A read that observes all effects of a given base table change also observes
  all effects of earlier changes to that base table that follow \emph{the same
  dataflow path}.
\end{invariant}

This invariant ensures that Noria does not drop updates, so that downstream
views ultimately reflect each base table change. The invariant is scoped to the
same dataflow path so Noria can process updates on parallel strands of dataflow
concurrently (\S\ref{s:parallel}). As long as Noria upholds the invariant on
each strand in isolation, no updates may be dropped anywhere.

Together these invariants ensure that Noria's views eventually reflect every
base table change exactly once. Each base table change triggers updates in the
dataflow, and by Invariant~\ref{i:no-holes} none of those updates can be dropped
if the system is to make progress. The ``at most once'' from
Invariant~\ref{i:no-spurious} must therefore mean that each base table change is
reflected exactly once if the system makes progress. And since Noria's operators
commute, as long as all updates are applied, the correct output must eventually
result.

Noria upholds these invariants without partial state, which is usually easy
barring implementation bugs with commutative operators\footnote{The join
consistency mechanism from \S\ref{s:join-state-dupe} is an example where Noria
must work specifically to uphold the invariants without partial state.}.
However, with partial state, they become harder to uphold as upquery response
may race with updates reflected within them. If Noria applies a given update as
it flows through the dataflow, that same update cannot also be applied as part
of an upquery response. And similarly, if an update is dropped, it must be
included with a later upquery response or it may never be applied.

% Note: this invariant is more an additional guarantee than a system invariant.
% While it is useful, it is very subtle (see the text), and not required for
% correctness. Therefore, it was scrapped.
%
% \begin{invariant}
%   \label{i:no-backsies}
%   A read issued at time $t$ and again at time $t' > t$ reflects the effects of
%   at least as many updates at $t'$ as at $t$.
% \end{invariant}
%
% This invariant ensures that \emph{if} the system makes progress, that progress
% persists\,---\,informally, Noria shouldn't expose the effects of an update, and
% then roll back those effects at a later time. Of course, if a later update
% happens to mask the effects of the earlier update, that is fine\,---\,the
% invariant is still preserved.
%
% This invariant is weaker than it first appears: since Noria is permitted to
% expose even incomplete effects of \emph{later} updates, the effects of an
% earlier update may be temporarily masked. For example, it \emph{is} possible to
% observe an output row introduced by an earlier update disappear while a future
% replacement is only partially processed. This violates the intuitive
% understanding of the invariant, but not its literal definition. The invariant is
% useful nonetheless: as the system makes progress, it ensures that more updates
% are reflected over time.

\subsection{Eviction}

Partial state also causes deltas that encounter missing state to be discarded
early in the dataflow (\S\ref{s:missing}). When this happens, the missing entry
may cause an update to be discarded even though downstream entries hold data
that would grow stale without that update. For example, consider what happens if
an operator counts the number of votes per author, and contains a count of 7 for
the author ``Jane''. Then, the state for ``Jane'' is evicted from some operator
upstream of the count. If an update now arrives at the operator where ``Jane''
is missing it would discard the update, and the downstream count would remain
perpetually stale, violating Invariant~\ref{i:no-holes}.

To avoid this situation, Noria must ensure that when an entry is evicted, any
dependent state downstream is also evicted:

\begin{invariant}
  \label{i:missing-suffix}
  If an operator encounters missing state while processing a record $r$ in an
  update, any downstream state that reflects $r$ must be evicted.
\end{invariant}

Upqueries generally uphold this invariant: as an upquery recurses up the
dataflow, it fills in missing state ``from the top down'' until the operator
that originally issued it receives the requested state. Thus, if a miss occurs
in the dataflow, that state is also missing downstream\footnote{Though not
always; see \S\ref{join-evictions}}.

When Noria evicts state in the middle of the dataflow, as described in
\S\ref{s:eviction}, this is no longer true. A miss mid-way down the dataflow no
longer implies that all related state is absent downstream. Therefore, to uphold
the property in the face of evictions, Noria issues evictions downstream
whenever it evicts entries from state in the middle of the graph. This ensures
that any future update that touches the evicted state can safely be discarded,
as any relevant downstream state has been discarded as well.

% Invariant~\ref{i:missing-suffix} implies the rule from earlier that partial
% state cannot exist upstream of non-partial state. Since state cannot be
% evicted from non-partial state, Noria would be unable to satisfy the invariant
% should it ever encounter missing state upstream of the non-partial state.

\subsection{Sharding}

For sharded base tables, the meaning of \emph{earlier} from
Invariant~\ref{i:no-holes} is unclear. In particular, concurrent updates to
different shards of the same base table do not happen before or after each other
in a well-defined way. This means that a read that observes the effects of one
base table change may not observe the effects of another change that happened
before if the other update went to a different base table shard. Effectively,
different shards constitute different dataflow paths in the definition of
Invariant~\ref{i:no-holes}.

The same applies for sharded views. Noria updates the view shards independently,
so if the effects of an update are present in one shard of a view, they may not
yet have manifested in another shard of that same view.

\section{Linear Dataflow}
\label{s:partial:linear}

Consider a single strand of dataflow, where each operator has at most one input
and at most one output. For partial state to be correct, it must be the case
that computing missing results with an upquery that combines all past deltas
into a single update produces the same results as processing the same deltas
one-at-a-time.

Recall that the deltas that flow through the dataflow represent changes to the
current state of the operator that emitted the delta. If a base table produces
a negative delta for a row $r$, it means that $r$ is no longer in that base
table's current state. An upquery fetches current state\,---\,the sum of all
past deltas emitted by the queried operator\footnote{As mentioned in
\S\ref{s:missing}, if the dataflow encounters missing state while processing an
update, it discards that update. This means that there may be holes in an
operator's state. If an upquery encounters such a hole, the dataflow fills the
hole with another upquery before proceeding.}\,---\,and feeds it through the
same chain of dataflow operators that individual deltas go through.

For upquery processing to be equivalent to one-at-a-time delta processing, it is
necessary that processing a combined update through all the dataflow operators
is equivalent to processing each of the combined updates through those same
operators. Or, more formally, with operators $f_1$ through $f_N$, and past
deltas $d_1$ through $d_M$:

\begin{eqnarray*}
  \sum^M_{i=1}\left(f_N \circ \dots \circ f_1\right)\left(d_i\right) = \
  \left(f_N \circ \dots \circ f_1\right)\left(\sum^M_{i=1}d_i\right)
\end{eqnarray*}

With a single operator, this trivially holds since all Noria operators are
distributive over delta addition:

\begin{eqnarray*}
  \sum^M_{i=1}f\left(d_i\right) = \
  f\left(\sum^M_{i=1}d_i\right)
\end{eqnarray*}

Using this property, and the fact that all operators produce and consume deltas,
it is possible to ``shift'' the delta sum across operator compositions:

\begin{eqnarray*}
  \sum^M_{i=1}\left(f_{n+1} \circ f_n\right)\left(d_i\right) &=& \sum^M_{i=1}f_{n+1}\left(f_n\left(d_i\right)\right) \\
  &=& f_{n+1}\left(\sum^M_{i=1}f_n\left(d_i\right)\right) \\
  &=& f_{n+1}\left(f_n\left(\sum^M_{i=1}d_i\right)\right) \\
  &=& \left(f_{n+1} \circ f_n\right)\left(\sum^M_{i=1}d_i\right)
\end{eqnarray*}

Therefore, the same ultimate state results whether the system executes each
dataflow operator in sequence on individual deltas, or whether it first sums all
the deltas into a single update, and then executes the operators in sequence
over that. Or, stated differently, if normal dataflow processing does not
violate the correctness invariants, the same must be true of processing a
combined upquery response.

Because the first three invariants do not deal with missing state, the argument
above concerns itself only with the processing of updates in the normal case.
However, the system must also uphold Invariant~\ref{i:missing-suffix}, which
dictates that Noria cannot discard messages that may affect non-missing,
downstream state. This is not captured by the argument above, but happens to be
the case for linear sequences of dataflow. Upqueries traverse the dataflow from
the leaves and up, and fill entries from the top down as the responses flow down
the dataflow. Thus, if some key $k$ is present in a materialization $m$, it must
also be present at every materialization above $m$ from the upquery chain that
ultimately produced the entry for $k$ in $m$. Since updates are discarded only
when they encounter missing state, a miss on $k$ anywhere in the dataflow
implies that $k$ is also absent downstream.

The system invariants are thus all upheld for any operator sequence.

\section{Diverging Dataflow}
\label{s:partial:diverging}

Dataflow graphs in real applications are rarely linear. They have branches where
the dataflow diverges, such as if two views both contain data from the same
table. When the dataflow diverges, upstream operators may receive multiple
upqueries for the same data. This happens if multiple downstream views encounter
missing entries that rely on the same upstream data.

The primary concern in this case is that the multiple upquery responses not
result in data duplication, and thus violate Invariant~\ref{i:no-spurious}. If a
stateful operator processes two upquery responses that both reflect some base
table row, the effects of that row would now be duplicated in the operator's
state.

Since upquery results only ever flow along the same edges that the upquery
followed on its way up the dataflow (\S\ref{s:upqueries}), such duplicates are
not a concern for materialization not on the upquery path. Those other branches
will never see the upquery response in the first place. Duplication is only a
concern for materializations that lie on the upquery path.

Section \ref{s:upquery:selection} noted that upquery paths are trimmed such that
they only reach back to the \emph{nearest} materialized state to the target.
Beyond improving efficiency, this is also important for correctness. It ensures
that there are no stateful operators on the upquery path between the source and
the destination. If there were, that operator's state would be used as the
upquery source instead. Since it is safe to process the same record through a
stateless operator multiple times, this ensures that the processing of the
upquery response on the path to the target state never duplicates effects.

Partial state on divergent dataflow thus upholds the system invariants.

\section{Merging Dataflow}
\label{s:partial:merging}

Most applications use joins or unions in their queries, which cause strands of
dataflow to combine. Such dataflow constructions introduce the possibility of
data races. Now, updates may arrive at an operator from two inputs at the same
time, and the operator may process either one before the other. Furthermore,
upqueries must now retrieve data from \emph{all} ancestors, and ensure that they
combine such that the system invariants are maintained.

How upqueries work across multi-ancestor operators depends on the semantics of
that operator. The only two relational multi-ancestor operators, unions and
joins, are discussed below.

\subsection{Unions}
\label{s:upqueries:union}

Unions merely combine the input streams of their ancestors, and includes little
processing beyond column selection. An operator that wishes to upquery past this
operator must therefore \emph{split} its upquery; it must query each ancestor of
the operator separately, and take the union of the responses to populate all the
missing state.

With concurrent processing, the multiple resulting responses may be arbitrarily
delayed between the different upquery paths, which can cause issues. Consider a
union, $U$, across two inputs, $A$ and $B$, with a single materialized and
partial downstream operator $C$. $C$ discovers that it needs the state for $k =
1$, and sends an upquery for $k = 1$ to both $A$ and $B$. $A$ responds first,
and $C$ receives that response.

$C$ must remember that the missing state is still missing lest it expose
incomplete state downstream. If it received an application read for $k = 1$, it
could not reply with \textbf{just} $A$'s state, as this might violate
Invariant~\ref{i:no-holes}. However, this alone is not sufficient to uphold the
system invariants.

Imagine that both $A$ and $B$ send one normal dataflow message each, and both
include data for $k = 1$. When these messages reach $C$, $C$ faces a dilemma. It
cannot drop the messages, since the message from $A$ includes data that was not
included in $A$'s upquery response. If it dropped them, those updates would be
lost, and results downstream would not be updated, violating
Invariant~\ref{i:no-holes}. But it also cannot apply the messages, since B's
message includes data that will be included in $B$'s eventual upquery response.
If it did, that data would be duplicated, which violates
Invariant~\ref{i:no-spurious}.

\begin{listing}
  \begin{minted}{python}
if is_upquery_response(d):
  buffered <- buffer[upquery_path_group(d)][key(d)]
  if len(buffered) == ninputs - 1:
    # this is the last upquery response piece.
    # emit a single, combined response
    emit(sum(buffered) + d)
    delete buffered
  else:
    # need responses from other parallel upqueries.
    buffered[from(d)] = d
    discard(d)
else:
  # this is a normal dataflow delta.
  # see if any changes in the delta
  # affect buffered upquery responses.
  for group_id, key_buffers in buffer:
    for change in d:
      change_key <- change[key_column(group_id)]
      # note the dependence on from(d) below.
      # changes from parents that have not produced
      # an upquery response yet are ignored; they
      # are represented in the eventual response.
      buffered <- key_buffers[change_key][from(d)]
      if buffered:
        buffered += change
  # always emit the delta, as other downstream
  # state may depend on it. any operator that is
  # waiting for missing state will discard.
  emit(d)
  \end{minted}
  \caption{Pseudocode for union buffering algorithm upon receiving a delta
  \texttt{d}. \texttt{buffer} starts out as an empty dictionary.
  \texttt{upquery\_path\_group} is discussed in the text.}
  \label{l:union-buffer}
\end{listing}

To mitigate this problem, unions must \textit{buffer} upquery results until
\emph{all} their inputs have responded. In the meantime, they must \emph{also}
buffer updates for the buffered upquery keys to ensure that a single, complete,
upquery response is ultimately emitted. Listing~\vref{l:union-buffer} shows
pseudocode for the buffering algorithm.

For unions to buffer correctly, they must know which upquery responses belong to
the ``same'' upquery. If there is only one upquery path through the union to
each ancestor, this is straightforward, as all upquery responses for a key $k$
are responses to the same upquery, and should be combined. However, in more
complex dataflow layouts, this is not always the case.

\begin{figure}[t]
  \centering
  \includegraphics{diagrams/Chained Unions.pdf}
  \caption{Chained unions. Only nodes $a$, $b$, and $v$ hold state. Highlighted
  in \textbf{\color{set1}blue} are two upquery paths that $\cup_1$ must combine
  upquery responses for.}
  \label{f:chained-union}
\end{figure}

Figure~\ref{f:chained-union} shows a dataflow segment where the precise grouping
mechanism is important (\texttt{upquery\_path\_group} in the code listing).
There are three unions in a chain, which makes eight distinct upquery paths. If
$v$ encounters missing state, it must therefore issue eight upqueries, one for
each path. $a$ and $b$ both appear as the root of four paths, and will be
upqueried that many times. The issue arises at the unions, which need to do
the aforementioned union buffering.

Ultimately, a single upquery response must reach $v$. This means that $\cup_3$
must receive two upquery responses, one from $e$ and one from $f$, which it must
then combine. So $\cup_2$ must \emph{produce} two upquery responses, one
destined for $e$ and one for $f$. This in turn means that $\cup_2$ must receive
two upquery responses from $c$, and two from $d$. Which again means that
$\cup_1$ must produce four responses, two for $c$ and two for $d$, out of the
eight responses it receives (four from $a$ and four from $b$).

These are all the upqueries that pass through $\cup_1$:
        
\begin{multicols}{4}
\begin{description}
  \item [1.] $a\,\to\,c\,\to\,e$
  \item [2.] $a\,\to\,c\,\to\,f$
  \item [3.] $a\,\to\,d\,\to\,e$
  \item [4.] $a\,\to\,d\,\to\,f$
  \item [5.] $b\,\to\,c\,\to\,e$
  \item [6.] $b\,\to\,c\,\to\,f$
  \item [7.] $b\,\to\,d\,\to\,e$
  \item [8.] $b\,\to\,d\,\to\,f$
\end{description}
\end{multicols}

$\cup_1$ must combine these so that each downstream union receives the responses
that they expect from their inputs. This grouping achieves that:

\begin{multicols}{2}
\begin{description}
  \item [1/5.] $a/b\,\to\,c\,\to\,e$
  \item [2/6.] $a/b\,\to\,c\,\to\,f$
  \item [3/7.] $a/b\,\to\,d\,\to\,e$
  \item [4/8.] $a/b\,\to\,d\,\to\,f$
\end{description}
\end{multicols}

The key observation is that the distinction between $a$ and $b$ does not matter
downstream of $\cup_1$; a delta that arrived from $a$ is indistinguishable from
one that arrived from $b$. Similarly, the distinction between $c$ and $d$ no
longer matters past $\cup_2$, and the same for $e$ and $f$ past $\cup_3$.
\texttt{upquery\_path\_group} is thus defined as a unique identifier for $v$'s
upquery plan plus the sequence of nodes between the union and the target of the
upquery response.

\subsection{Joins}

Upqueries across unions must go to all the ancestors. But across joins,
upqueries must only go to \textbf{one} ancestor. This is because a join that
processes a message from one ancestor already queries the ``other'' ancestor and
pulls in relevant state from there. If both sides were queried, the processing
of the upquery responses at the join would produce duplicates of every record.

Noria supports two types of joins: inner joins and partial outer joins (i.e.
``left'' and ``right'' joins). For an inner join, either ancestor can be the
target of the upquery, whereas for a partial outer join, the upquery \emph{must}
go to the ``full'' side\,---\,the side from which all rows are
yielded\footnote{An upquery for a column that originates from the non-full
ancestor must be routed according to its value. If it is \texttt{NULL}, the
upquery must go to the left, with additional later filtering, whereas if it is
non-\texttt{NULL}, it can go to the non-full side without issue. Noria does not
support such upqueries.}. Otherwise, the upquery may produce only a subset of
the results for the join.

\subsubsection{Dependent Upqueries}

Since upqueries travel through only one ancestor of a join, joins do not need to
buffer upquery responses the same way unions do. However, when a upquery
response passes through a join operator, the join does perform lookups into the
state of the other side of the join. With partial state, those lookups may
themselves encounter missing entries. When this happens, a problem arises: Noria
\emph{must} produce an downstream upquery response because the application is
waiting for it, but cannot produce that response since required state is
missing.

For the purposes of exposition, and without loss of generality, the text below
refers to the join ancestor that was upqueried as the left side, and the
ancestor that a lookup missed in as the right side.

The join must issue an upquery to the right hand side for the state that is
missing to complete the processing of the original upquery response from the
left. However, this \textit{dependent upquery} may take some time to complete,
and the system must decide what to do in the meantime. Recall that the join is
still in the middle of processing an upquery response.

An obvious, but flawed strategy is to have the join block until the response
arrives. This would not only stall processing of deltas from the left parent,
but also leads to a deadlock. In order to observe the eventual upquery response,
the join's domain must continue to process incoming messages to the right parent
(\S\ref{s:join-state-dupe}). But in doing so, it may encounter a different
upquery response from the right parent. That upquery response may require a
lookup into the left parent's state, which may itself encounter missing entries.
The join is then forced to block on both inputs perpetually.

Instead, the join \emph{discards} the current upquery response, and remembers
the upquery parameters that triggered it, and the missing state that must be
filled. It then continues processing the next update as normal. When the missing
entries are eventually filled, Noria \emph{re-issues} the original upquery to
the join's left parent using the saved parameters. This time, all entries
required for the lookups into the right-hand parent's state are present, and the
downstream upquery response can be produced. As far as the downstream dataflow
is concerned, nothing abnormal has happened\,---\,the upquery response just took
longer to arrive.

\subsubsection{Incongruent Joins}
\label{join-evictions}

As discussed in \S\ref{s:partial:linear}, if some key $k$ is present in a
materialization $m$, it is also present at every materialization above $m$ from
the upquery chain that ultimately produced the entry for $k$ in $m$. However,
certain query graphs produce dataflow where more than one key is used to compute
an entry. Consider a dataflow that joins two inputs, \texttt{story} and
\texttt{user}, on the story's author field. A downstream operator issues an
upquery for story number 7. The upquery is issued to \texttt{story}, which
produces a message that contains story number 7 with author ``Elena''. That
message arrives at the join, which issues a dependent upquery to \texttt{user}
for ``Elena''. When that dependent upquery resolves, the join produces the final
upquery response, and the state for story number 7 is populated in the
downstream materialization.

Next, an editor changes the author for story number 7 to ``Talia''. This
takes the form of a delta with a negative multiplicity record for \texttt{[7,
"Elena"]} and a positive one for \texttt{[7, "Talia"]}. When this delta arrives
at the join, it may now miss when performing the lookup for ``Talia''. According
to the partial model so far, the join should drop \texttt{[7, "Talia"]}, and
only allow the negative for ``Elena'' to propagate to the downstream
materialization. But this violates Invariant~\ref{i:missing-suffix}, since there
exists downstream state that reflects the discarded update. And indeed, when
this happens, the state for article number 7 becomes empty (though not missing),
and any subsequent read for article number 7 receives an empty response, which
violates Invariant~\ref{i:no-holes}.

What happened here was that the entry for key $k$ in the leaf-most
materialization depends not only on state entries indexed by the same $k$
upstream, but also on state entries indexed by other keys upstream. While $k$
must be present upstream, no such guarantee exists for other keys.

This is a result of \textit{incongruent joins}; joins whose join column is not
the same as the downstream key column. Incongruence is determined with respect
to each upquery path that flow through a join. In the case above, the author
join is incongruent with an upquery on the story number column, since the join
column is the author column. However, the join is congruent with upqueries from
a hypothetical downstream view that is keyed by author instead. A join that is
incongruent with any upquery path that flows through it is considered an
incongruent join.

Noria can easily recognize incongruent joins through key provenance
analysis\,---\,if an upquery flows through a join, and the upquery column is not
the same as the join column, the join is incongruent. If an incongruent join
encounters missing state while processing a delta at runtime, it must take
action to ensure that downstream state remains correct. Since the domain that
processes the join cannot produce a valid delta, and does not know what state is
present and missing in the downstream dataflow, its only option is to issue an
eviction for any downstream state that \emph{may} be rendered stale. Concretely,
if an incongruent join processes a record $r$ and encounters a missing state
entry, it should issue an eviction downstream on all incongruent upquery paths
using the appropriate values from $r$. For example, if the join column is $c_j$,
and upquery path $u_i$ through the join is keyed by column $c_i$, then the join
should issue downstream evictions of $r[c_i]$ for each $u_i$ where $c_i \neq
c_j$.

\paragraph{All Together Now.}
%
With unions and joins covered, the argument is complete. In all dataflows that
Noria can construct, no matter how they diverge and merge, the outlined
mechanisms ensure that the system invariants are maintained at every node, and
thus in Noria as a whole.

\section{Sharding}
\label{s:challenge:sharding}

Noria supports sharding cliques of operators to add parallelism to particular
sections of the dataflow (\S\ref{s:noria:sharding}). When Noria decides to shard
operators in this way, upqueries must continue to work. Partial state with
sharding mainly follows the rules of partial across unions
(\S\ref{s:upqueries:union}), with three changes:

First, if the node that receives the upquery, R, is sharded the same way as the
querying node, Q, the upquery is sent \emph{only} to the same shard of R as the
one that is querying. This is called a \textit{narrow} upquery, and avoids
queries to shards that hold no relevant data. This rule applies even if the
upquery key differs from the sharding key, since while other shards may have
relevant data, that data would be discarded before reaching the current node
anyway. Noria decides whether upqueries should be narrow or broad when it
determines an operator's upquery plan\,---\,key provenance provides sufficient
information to make the decision.

Second, when a narrow upquery response reaches the \emph{first} shard merger
(effectively a union across shards) on its path, the response must not be
buffered, unlike other upquery responses across unions. This is because
the other upstream shards will not be sending responses.

Third, when the upquery response for an upquery that originated at a sharded
node reaches the \emph{last} sharder on its path, that sharder must direct that
response only to the querying shard. This is equivalent to the general rule that
upquery responses only flow along the edges that the upquery traversed. The
upquery that triggered the response did not touch other shards of the upquery
originator, and so the response should not go there.

Beyond those three modifications, the existing logic for handling upqueries
across forked strands of dataflow is sufficient.

% \footnote{Since shard mergers have only one operator (but many shards) as their
% ancestor, they use the shard index of the operator that sent each input as a way
% to distinguish between input paths, rather than the index of the operator
% itself. Otherwise, they operate entirely as a union.}
