Noria without partial state, as described in \S\ref{s:noria}, uses significant
amounts of memory. All results for all queries must be materialized, and unlike
traditional caching approaches, unimportant cached results are not evicted to
free up memory. To address the high memory use of traditional materialized
views, this thesis proposes \textit{partially materialized state}, often
shortened to partial state. Partial state enables Noria to store and maintain
only a subset of a materialized view's contents, and to compute missing state on
demand. Partial state also enables Noria to implement eviction, so that the
materialization cost is kept low even as the underlying workload changes.

This chapter discusses the partially stateful model and its components. The next
chapter examines the practical challenges that arise when partial state is
implemented in a dataflow system.

\section{Missing State}
\label{s:missing}

Partial state allows state to be \textit{missing}. Missing state indicates that
a particular value is not yet known, and must be computed on demand if the
application queries for it. State can be marked as missing both in state that is
internal to the dataflow, like the state of an aggregation, and in externally
visible state like Noria's query result caches.

With partial state, most Noria state starts out as missing, and is populated
according to what data the application queries for. This also allows Noria to
quickly adopt new views, since in the common case no computation need happen
when additional operators are added.

\begin{figure}
  \centering
  \includegraphics{diagrams/Indexing.pdf}
  \caption{Multiple indexes over the state in a single view in Noria. Even
  though \emph{some} rows for author B are \textbf{\color{set3}present}, some
  are \textbf{\color{set2}missing} and so the entry for B is considered missing
  in the \texttt{author} index. Even though no rows for author C are present,
  the index entry is not marked as missing, which would happen if Noria has
  checked that there are indeed no rows for author C.}
  \label{f:indexing}
\end{figure}

Missing state manifests as missing entries in particular indices. Indexes over a
given state are either all partial or none of them are. This may seem strange
based on how indices work in traditional relation databases.
Figure~\vref{f:indexing} gives an example with two different partial indices
over a view that holds a unique story identifier and the story's author. One
index is over the primary key column \texttt{id}, and one is over the story
author. Here, even though some rows with author B are present, the index entry
for author B is still considered missing, since not \emph{all} rows with author
B are present. This is necessary, as otherwise an application query for stories
authored by B would return a result with missing rows, which violates
Invariant~\ref{i:no-holes}.

While there are no rows for authors C or D, C is considered complete because
Noria has checked upstream that there are indeed no stories written by C. For D,
Noria has not yet done an upstream check, and therefore does not know what the
true result set is.

If Noria encounters missing state while processing an update, the update must
not affect query results that the application has indicated interest in. In such
a case, Noria has two options: eagerly compute the missing state before
proceeding, or discard the update. To avoid unnecessarily maintaining
unimportant cached results, Noria drops updates in this case.

An important corollary of the above is that partial state must be enabled
on all stored state \emph{below} any partial state. It is illegal for the
dataflow to contain state for two nodes $A$ and $B$ where $A$ is an ancestor of
$B$, $A$ uses partial state, and $B$ does not use partial state. To see why,
consider what would happen if an update arrives at $A$ for a missing entry. $A$
would discard that update, and $B$'s state would never reflect that update even
though it may reflect later updates, which violates
Invariant~\ref{i:no-holes}.

\section{Upqueries}
\label{s:upqueries}

If an application requests data that is found to be missing, Noria issues an
\textit{upquery} to compute the requested data. Upqueries flow ``up'' the
dataflow graph, towards the base tables at the ``top'', and constitute a request
for the target of the upquery to replay past data. Upqueries may recurse if the
requested state is not available at the initial target.

The response to an upquery takes the form of a regular dataflow update that
flows down the dataflow. It combines all past deltas pertinent to the upquery
into a single update, and holds only positive deltas that represent the current
set of relevant records.

Operators are not generally aware if they are processing an update that resulted
from an upquery response. The upquery response flows in-line with other dataflow
updates, and follows the edges of the dataflow. However, upquery responses are
special in two key ways. First, they only propagate along edges towards the
operator that issued the upquery, so that one upquery does not populate the
relevant data in the state of \emph{every} operator. And second, if an operator
encounters missing state while processing an upquery response update, it does
\emph{not} discard that update as it would a regular dataflow update. Instead,
it does the work necessary to fill the missing state and then process that
update.

When an application query encounters missing state in a view, Noria needs to
know what upqueries to issue to fill that state. The set of upqueries for each
view is that view's \textit{upquery plan}. Noria determines upquery plans by
analyzing each view's query when the application first installs that view, and
deciding how best to recompute its results. It does so by finding all
\emph{possible} upquery plans, choosing among them, and then informing all
involved domains of the chosen plan. There may be multiple possible candidates
if there are multiple equivalent ways to compute the missing state, such as by
changing the direction in which joins are executed as explained below.

\subsection{Key Provenance Tracing}

To determine what upqueries can reconstitute missing entries in a given index,
Noria must trace the view's parameter column (the \texttt{?} in the query) back
to a column in upstream state. The intuition here is that in order to answer the
application's query of ``give me the results where column $C$ has value $x$'',
Noria must be able to replay rows where $C = x$ from somewhere. Or, phrased
differently, when the output for $C = x$ is missing, Noria must have a way to
get the inputs that \emph{generate} $C = x$. As an example, if a view counts
books by a given author, and the current count for author $a$ is missing, Noria
must be able to somehow produce all books by author $a$.

\begin{figure}[t]
  \centering
  \includegraphics{diagrams/Key Provenance.pdf}
  \caption{Key provenance for each column in the \texttt{StoriesWithVC} view
  from Listing~\ref{l:vote-src}. Notice that \texttt{\color{set3}story\_id} has
  multiple base table origins, and \texttt{\color{set2}vcount} does not trace
  back to any base table columns. The query only uses
  \texttt{\color{set3}story\_id} as a parameter, so only its provenance is used
  to choose the upquery path.}
  \label{f:key-prov}
\end{figure}

More generally, in order to recompute the results where $C = x$ in some view
$V$, Noria must determine the \textit{key provenance} of $C$; where $C$ ``came
from''. Noria computes key provenance by tracing columns ``up'' the dataflow to
where they originate, which results in a \textit{provenance graph}.
Figure~\ref{f:key-prov} shows the provenance graph for the
\texttt{StoriesWithVC} view from Listing~\vref{l:vote-src}, and illustrates two
important properties of key tracing:

\begin{enumerate}
  \item An output column may trace to multiple input columns if it corresponds
    to the join column in a join, or if it passes through a union. The
    provenance of the \texttt{story\_id} column, for example, traces both to
    \texttt{stories.id} and \texttt{votes.story\_id}.
  \item An output column may be entirely computed, and thus have no association
    with a column in the inputs. The \texttt{vcount} column is computed by the
    \texttt{VoteCount} aggregation, and does not exist in the input data.
\end{enumerate}

In Listing~\ref{l:vote-src}, Noria is asked to parameterize
\texttt{StoriesWithVC} by the \texttt{story\_id} column. The key provenance
graph tells Noria that it can request input data for a given \texttt{story\_id}
by sending an upquery either to the \texttt{stories} table using the \texttt{id}
column, or to the \texttt{votes} table using the \texttt{story\_id} column.

\paragraph{Broken Provenance.}
Consider what would happen if Listing~\ref{l:vote-src} had \texttt{WHERE vcount
= ?} as its parameter instead. If an application query misses in that case, the
upquery would have to be sent to \texttt{VoteCount}, and query for ``all stories
whose vote count is $x$''. If that state is present, all is well, but if
\texttt{VoteCount} is missing the state for \texttt{vcount = x}, there is a
problem: Noria has no way to compute the missing state except by replaying
\emph{all} state in \texttt{votes} without using an index. This is equivalent to
a full table scan in a traditional database. Noria's only%
%
\footnote{Noria cannot disable partial state just for \texttt{StoriesWithVC},
since that would place a partial index above a non-partial index.}
%
efficient option is to disable partial
state for \texttt{VoteCount}. This ensures that any upquery to it never misses,
and therefore a table scan is never needed. Instead, the table scan is performed
only once: when the view is initially added. But this comes at the cost of
maintaining the entire result set of the query for all parameter values.

\paragraph{Asymmetric Provenance.}
The join in Listing~\ref{l:vote-src} is an inner join ($\bowtie$), so Noria can
upquery \emph{either} side. If it upqueries the ``left'' side of the join,
normal forward processing performs the necessary lookups into the ``right'' side
of the join, and vice-versa. However, if the query used a left or right
\emph{outer} join, Noria must upquery a particular side of the join. For a left
join, it must upquery the left ancestor, or risk missing rows in the left
ancestor that have no matching rows in the right ancestor. This would result in
those rows never appearing in downstream views, which violates
Invariant~\ref{i:no-holes}. For a right join, the same logic applies, but
mirrored to the right ancestor.

\paragraph{Disjoint Provenance.}
If the provenance of a column crosses a union, \emph{all} ancestors of that
union must be upqueried, not just one as is the case with upqueries through a
join. Unlike with a join, the regular dataflow processing of the upquery
response through a union does not bring along results from the other ancestors,
so the requesting operator must ask them individually.

\subsection{Path Selection}
\label{s:upquery:selection}

Once Noria has obtained a set of candidate upquery paths through key provenance,
it must decide on an upquery plan based on those paths. If there is only one
candidate, the choice is trivial. But with symmetric joins, multiple candidate
paths may be generated. Here, Noria is free to use whatever heuristics it sees
fit to pick which side of the join to send upqueries to. For example, it may
choose to send upqueries to the larger of the joined inputs so that fewer
lookups are needed when processing the response.

Key provenance tracing produces upquery paths that reach all the way back to the
origin of a column, which is usually located at the base tables. However, it
would be inefficient for operators to issue upqueries all the way to the base
tables on every miss. Some intermediate state may already have the necessary
data, and the upquery data could be sourced from there instead. Noria therefore
trims the paths from key provenance such that only the suffix of operators
starting at the last materialized state are included. For example, in
Figure~\ref{f:key-prov}, if Noria decides to upquery \texttt{StoriesWithVC}
through \texttt{VoteCount}, the upquery path would source its data from
\texttt{VoteCount}, not from \texttt{votes}.

If an upquery reaches its origin and finds that the requested state is missing
there too, a second upquery is issued using the origin's upquery paths, and only
when that upquery resolves does the original upquery resume. Upqueries may
recurse all the way up to the base tables this way, but avoid doing so if any
intermediate state can be re-used.

This process leaves Noria with a set of paths to upquery when it encounters a
missing entry. In many ways, the procedure is similar to that of traditional
query planning, and some techniques from there could likely be applied. At the
same time, upquery planning introduces some unique challenges. First, planning
cannot change the order of existing operators, since they are part of the
running dataflow that is already maintaining other views. To modify them, Noria
would have to stop the dataflow to rewire the edges. Second, upquery plans still
rely on forward incremental dataflow to compute the final results\,---\,a join
strategy that cannot be executed incrementally is no good, no matter how well it
might perform.

Once Noria has a plan, that plan is communicated to all domains that appear
along each path in the plan. This is necessary so that each domain knows where
to route upquery responses that are part of a given plan, and does not
disseminate the response to the entire downstream dataflow.

\subsection{Index Planning}

When an upquery arrives at the materialization it wants to source data from,
Noria needs an efficient way to find the requested data. Specifically, Noria
needs an index on the materialization whose key matches the lookup key of the
upquery. Therefore, when Noria announces the upquery plan, it may also add
additional indices to existing state to facilitate efficient execution of the
new upqueries. In this way, upquery plans adds additional indexing obligations
that Noria must take into account.

The key provenance information from Figure~\ref{f:key-prov} gives Noria all the
information it needs to set up these indexes: an index is needed on the upquery
key column on each state on the chosen upquery paths. In the case of the view
from Listing~\ref{l:vote-src}, an index is needed on
\texttt{StoriesWithVC.story\_id}, as well as either \texttt{stories.id} or both
\texttt{VoteCount.story\_id} and \texttt{votes.story\_id}\footnote{An index is
needed on \texttt{votes.story\_id} since the upquery to \texttt{VoteCount} may
recurse.}, depending on which upquery path Noria chooses across the join.

\section{Eviction}
\label{s:eviction}

Over time, the subset of data that the application cares about tends to change.
When it does, query results that were accessed previously may no longer be
important to maintain as they are no longer accessed. Partial state allows Noria
to cater to such changing application patterns by \textit{evicting} state
entries after they have been computed. When an entry is evicted, it is marked as
missing, and subsequent requests for that state trigger an upquery as usual for
missing state.

When Noria evicts an entry from state in the middle of the dataflow, the missing
entry may cause an update to be discarded even though downstream entries hold
data that would grow stale without that update. For example, consider what
happens if an operator counts the number of votes per author, and contains a
count of 7 for the author ``Jane''. Then, the state for ``Jane'' is evicted from
some operator upstream of the count. If an update now arrives at the operator
where ``Jane'' is missing it would discard the update, and the downstream count
would remain perpetually stale, violating Invariant~\ref{i:no-holes}.

To avoid this situation, Noria must ensure that when an entry is evicted, any
dependent state downstream is also evicted:

\begin{invariant}
  \label{i:missing-suffix}
  If an operator encounters missing state while processing a record $r$ in an
  update, any downstream state that reflects $r$ must be evicted.
\end{invariant}

Upqueries as described so far, without evictions, generally uphold this
invariant\footnote{Though not always; see \S\ref{join-evictions}}. As upqueries
recurse up the dataflow, they fill in missing state ``from the top down'' until
the operator that originally issued the upquery receives the requested state.
Misses that occur while processing an upquery causes recursive upqueries, which
fill that missing state before proceeding.

To uphold the property in the face of evictions, Noria issues evictions
downstream whenever it evicts entries from state in the middle of the graph.
This ensures that any future update that touches the evicted state can safely be
discarded, as any relevant downstream state has been discarded as well.

Invariant~\ref{i:missing-suffix} implies the rule from earlier that partial
state cannot exist upstream of non-partial state. Since state cannot be evicted
from non-partial state, Noria would be unable to satisfy the invariant should it
ever encounter missing state upstream of the non-partial state.
