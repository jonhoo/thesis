\documentclass[fontsize=12pt,paper=letter,draft=true]{scrbook}

% for 1.5 line spacing
\usepackage{setspace}
\onehalfspacing
% single spacing for table of contents
\AfterTOCHead{\singlespacing}

% recompute page layout based on the above
\recalctypearea

% more colors (like RedOrange)
\usepackage[dvipsnames]{xcolor}

% brewer qualitative colors
\definecolor{brewergreen}{HTML}{1b9e77}
\definecolor{brewerorange}{HTML}{d95f02}
\definecolor{brewerpurple}{HTML}{7570b3}

% so we can splice in PDFs
\usepackage{pdfpages}

% set up bibliography
\usepackage[
  backend=bibtex,
  sortcites=true,
  sorting=anyt,
  abbreviate=false,
  style=numeric,
  citestyle=numeric,
  isbn=false,
  url=false,
  doi=false]{biblatex}
\addbibresource{bibliography.bib}

% enumerate* and itemize*
\usepackage[inline]{enumitem}

% for \begin{comment}
\usepackage{verbatim}

% for source-code listings
\usepackage[newfloat,draft=false]{minted}

% for formulas
\usepackage{mathtools}

% to split lists into multiple columns
\usepackage{multicol}

% for "on page NN" reference
\usepackage[nospace]{varioref}

% for \sfrac
\usepackage{xfrac}

% do not reset page numbers at \mainmatter
\let\mainmatterorig\mainmatter
\renewcommand\mainmatter
 {\edef\p{\arabic{page}}%
  \mainmatterorig
  % we need to compute the actual current page number. we know the page number
  % from _before_ we called \mainmatter. but what is it now? well, it is
  % certainly that +1. but we also need to account for the next chapter starting
  % on a "right" (odd) page. we do this by adding the page number modulo two.
  % TODO: double check before final version
  \setcounter{page}{\p+1+(\p-\p/2*2)}%
 }

% an environment for todos
\newenvironment{inprogress}
  {\vspace{.5em} \color{brewerorange} \noindent \textbf{TODO}}
  {\vspace{.5em}}

% a command to indicate current editing progress
\newcommand{\resume}{
  \begin{center}
    \color{brewerorange}
    \hrule
    \vspace{1pt}
    \hrule
    \hrule
    \vspace{10pt}
    \textbf{This section is not yet complete.}
    \vspace{10pt}
    \hrule
    \hrule
    \vspace{1pt}
    \hrule
  \end{center}
}

% an environment for invariants
\newcounter{invn}
\renewcommand{\theinvn}{\Roman{invn}}
\newenvironment{invariant}
  {\vspace{.5em} \color{brewerorange} \refstepcounter{invn} \noindent \textbf{\color{brewerorange} Invariant \Roman{invn}.}}
  {\vspace{.5em}}

% for handy reference
%
% paragraph without spacing:
% \setparsizes{0pt}{0pt}{0pt plus 1fil}

% additional hyphenation rules
\hyphenation{da-ta-flow}

% in thesis: titlehead, subject, title, subtitle
\title{Noria: Partial State in Dataflow}
\author{Jon Gjengset}
\begin{document}

\frontmatter

% always arabic page numbering (default is roman in \frontmatter)
\pagenumbering{arabic}

\includepdf[pages=-]{./titlepage.pdf}
\leavevmode\thispagestyle{empty}\newpage % since title page is single-sided

\includepdf[pages=-]{./abstract.pdf}
\leavevmode\thispagestyle{empty}\newpage % since abstract is also single-sided

\section*{Prior Publication}
Much of this thesis was previously published in a conference paper~\cite{noria},
and represents the joint work of the coauthors of that paper.
\newpage

\section*{Acknowledgements}
\begin{spacing}{1}
  % TODO
  Goes here.
\end{spacing}

\tableofcontents

\mainmatter

\chapter{Introduction}

This dissertation proposes a practical system that lowers latency and increases
sustainable throughput for read-heavy web applications by automatically
maintaining a cache of the application's query results. Where traditional
databases re-execute queries each time they are issued, the presented system
stores and maintains a cache of past query results, and uses that to serve
queries more efficiently. As opposed to existing state-of-the-art systems, the
dissertation work supports building the cache on demand, and is able to evict
cache entries in response to a shifting workload.

Experimental results suggest that the presented work increases supported
application load by up to $20\times$ over MySQL, and reduces memory use by up to
$\sfrac{2}{3}$ compared to existing approaches.

\section{Motivation}

Most modern web applications share similar traits. They are
\textbf{interactive}: each page load has a user waiting on the other end. They
are \textbf{read-heavy}: most user interactions consume content rather than
produce new content. And they experience \textbf{significant skew}: a small
number of people, posts, creators, teams, and discussions make up the bulk of
interactions.

Such applications are usually poorly served by the traditional relational
databases that most of them use to store and query their underlying data.
These application tend to issue the same read-only queries again and again, with
the underlying data changing only infrequently. Existing databases do
not optimize for this kind of workload: they run each query in isolation, and
thus re-do work that has already been done many times over.

As a result, application authors often resort to ad-hoc, error-prone
techniques~\cite{ad-hoc-caching} to exploit their applications' workload
patterns and satisfy impatient users. They de-normalize their database schemas
by placing and updating computed values in the database tables, or introduce
key-value stores that \textit{cache} the result of expensive queries. All these
techniques introduce significant application complexity: the application authors
must include logic to ensure that the auxiliary derived state remains up to date
as the underlying data changes, that clients do not all flood the database when
results are not available in the cache, and that concurrent access to the
database and the cache never leaves the system in an inconsistent state.

\section{Existing Solutions}

Existing systems from industry~\cite{facebook-memcache, tao, flannel} and
academia~\cite{txcache, cachegenie, casql-consistency-thesis, pequod} have
attempted to chip away at this problem, but are often lacking in important ways.
Some require significant developer effort, and are infeasible to implement for
any but the largest companies. Some support only a restricted set of application
queries, or only provide infrastructure for developers to implement caching
themselves. Many maintain cached results only by evicting old results, and
cannot updating existing results in-place, which is wasteful.

% Noria replaces caching \emph{logic}, not caching just caching \emph{systems}.

Eons ago, the database community produced \textit{materialized views} as a
potential answer to the slow query problem. Materialized views store the results
of \textit{views} (i.e., named queries) with the goal of making those queries
faster to execute~\cite{materialized-views}. These materialized views can then
be maintained incrementally rather than re-executing queries when the underlying
data changes~\cite{materialized-survey}. However, few commercially available
databases support materialized views, and the ones that do come with significant
restrictions~\cite{mssql-materialized-view-restrictions}.

State-of-the-art research systems support flexible materialized
views~\cite{dbtoaster,materialize}, but do not support eviction or low-latency
reads\footnote{In these systems, reads cannot access the materialized view
directly, and must synchronize with the write-processing pipeline to extract
query results.}, and thus function poorly as a replacement for a cache. Many
recent state-of-the-art materialized view systems are also restricted to a
pre-declared set of queries, and cannot incorporate changes to the application
query set without restarting.

\section{Approach: Partial State}

This thesis proposes \textit{partial state} in Noria, a state-of-the-art
materialized view system that is already optimized for read-heavy, dynamic web
applications~\cite{noria}. Partial state enables Noria to materialize results
for queries on-demand and evict results when they are no longer useful.

Noria uses \textit{dataflow} to maintain its materialized views, a system
architecture that allows fast and distributed computation over a stream of data
changes. It represents computational dependencies as a directed acyclic graph
where edges represent data dependencies, and vertices represent computations
(like aggregations or joins) over the data that arrives over the incoming edges.
The work in this thesis adds support for partial state to Noria's dataflow so
that most of Noria's existing systems can be re-used.

Partial state allows state in Noria's materialized views to be marked as
\textit{missing}, and introduces \textit{upqueries} as a mechanism to compute
such missing state on-demand. Upqueries re-use the existing dataflow operators
that maintain the materialized views, and can theoretically be retrofitted onto
any existing dataflow system.

\section{Contributions}

The main contributions of this thesis are:

\begin{itemize}
 \item A model for, and implementation of, partial state in a dataflow-based
   materialized view system.
 \item An algorithm for implementing upqueries to populate missing state on
   demand.
 \item An analysis of the issues that arise when introducing partial state to a
   distributed, high-performance stateful dataflow processing system.
 \item Techniques for overcoming those issues while preserving system
	 correctness, performance, and scalability.
 \item Micro and macro evaluations of the performance and memory impact of
	 introducing partial state to dataflow.
\end{itemize}

\paragraph{Dissertation Outline.}

The rest of the dissertation is organized as follows: Section~\ref{s:noria}
describes the Noria dataflow system. Section~\ref{s:partial} introduces the
partially stateful dataflow model. Section~\ref{s:correct} describes additional
mechanisms that are needed to ensure that partially stateful dataflow produces
correct query results. Section~\ref{s:eval} evaluates Noria's implementation of
partial state on a realistic application query workload. Section~\ref{s:related}
explores related work. Section~\ref{s:disc} discusses shortcomings of, and
alternatives to, partial state. Finally, Section~\ref{s:future} outlines future
work on partial state.

For readers that are unfamiliar with database queries, materialized views,
dataflow, and application caching, but would still like to understand roughly
what this thesis is about, Appendix~\ref{s:simple} starting on
page~\pageref{s:simple} is for you.


\chapter{Background \& Related Work}


\section{Materialized Views}

\section{Caching Systems}

\section{Dataflow}

\section{Other Related Work}

\begin{comment}
http://www.vldb.org/pvldb/vol13/p1793-mcsherry.pdf
https://www.inf.ufrgs.br/prosoft/publications/2016/mertz-tse-2016-pre-print.pdf
https://people.csail.mit.edu/nickolai/papers/gupta-cachegenie.pdf
https://arxiv.org/abs/2004.14471
https://arxiv.org/abs/2001.00888
\end{comment}

Differential dataflow~\cite{naiad,differential-dataflow}, and its instantiation
in the commercial product Materialize~\cite{materialize}, bears a striking
resemblance to Noria at first glance. In particular... Also, see discussion
section on emulating partial state.


\chapter{Noria}
\label{s:noria}

In this thesis, partial state is implemented in Noria~\cite{noria}, a stateful,
dynamic, parallel, and distributed dataflow system designed for the storage,
query processing, and caching needs of typical web applications. Because of the
strong connection between Noria and partial state, this chapter describes the
design and implementation of Noria in sufficient depth to follow the remainder
of the thesis.

\section{High-level Overview}

\begin{listing}[h]
  \begin{minted}{sql}
/* base tables */
CREATE TABLE stories
  (id int, title text);
CREATE TABLE votes (story_id int, user int);
CREATE TABLE users (id int, username text);
/* internal view: vote count per story  */
CREATE INTERNAL VIEW VoteCount AS
  SELECT story_id, COUNT(*) AS vcount
    FROM votes GROUP BY story_id;
/* external view: story details */
CREATE VIEW StoriesWithVC AS
  SELECT id, author, title, url, vcount
    FROM stories
    JOIN VoteCount ON VoteCount.story_id = stories.id
   WHERE stories.id = ?;
  \end{minted}
  \caption{Noria program for a key subset of the Lobsters news
           aggregator~\cite{lobsters} that counts users' votes for stories.}
  \label{f:vote-src}
\end{listing}

Noria runs on one or more multicore servers that communicate with clients and
with one another using RPCs. A Noria deployment stores both \emph{base tables}
and \emph{derived views}. Roughly, base tables contain the data typically stored
persistently, and derived views hold data an application might choose to cache
to improve performance. Compared to conventional database use, Noria base tables
might be smaller, as Noria derives data that an application may otherwise
pre-compute and store denormalized in base tables for performance. Views, by
contrast, will likely be larger than a typical cache footprint, because Noria
derives more data, including some intermediate results. Noria stores base tables
persistently on disk, either on one server or sharded across multiple servers,
but stores views in server memory.

Noria's interface resembles parameterized SQL queries. The application supplies
a \emph{Noria program}, which registers base tables and views with parameters
supplied by the application when it retrieves data. Figure~\ref{f:vote-src}
shows an example Noria program for a news aggregator application where users can
vote for stories (\texttt{?} is a parameter). The Noria program includes base
table definitions, \emph{internal} views used as shorthands in other
expressions, and \emph{external} views that the application later queries. It
can be thought of as an extended schema for the application that includes its
queries. Noria supports much, but not all, SQL syntax.

To retrieve data, the application supplies Noria with an external view
identifier (e.g., \texttt{StoriesWithVC}) and one or more sets of parameter
values. Noria then responds with the records in the view that match those
values. To modify records in base tables, the application performs insertions,
updates, and deletions, similar to a SQL database. Data is represented as
structured records in tabular form~\cite{spanner, bigtable}.

Noria differs substantially from traditional databases in how queries are
executed. Rather than compute a query's results on demand when the
application executes it, Noria does so when the query view is defined.
Noria then caches, or \emph{materializes}, the contents of that view, and serves
queries to that view directly from that cache. To keep the materialized current,
Noria internally instantiates a dataflow program to continuously process the
application's writes and update the views as appropriate.

This kind of view materialization makes Noria particularly well-suited for
read-heavy applications that tolerate eventual consistency, since it shifts
query execution cost from reads, which are frequent, to writes, which are
infrequent. Noria focuses entirely on relational operators, rather than the
iterative or graph computations that are the focus of other dataflow
systems~\cite{naiad, differential-dataflow}.

The application may change its Noria program to add new views, to modify or
remove existing views, and to adapt base table schemas. Noria expects such
changes to be common and aims to complete them quickly, without restarting the
dataflow engine.

\section{Dataflow Execution}

To keep its materialized views from growing stale as the underlying data
changes, Noria uses dataflow. Noria compiles all the application queries into a
joint dataflow program, which it routes all application writes through. The
dataflow is a directed acyclic graph of relational operators such as
aggregations, joins, and filters. Base tables are the roots of this graph, and
external views form the leaves. Noria extends the graph with new base tables,
operators, and views as the application adds new queries.

When an application write arrives, Noria applies it to a durable base table and
injects it into the dataflow as an \emph{update}. Operators process the update
and emit derived updates to their children; eventually updates reach and modify
the external views. Updates are \emph{deltas}~\cite{roll, differential-dataflow}
that can add to, modify, and remove from downstream state. Deltas are similar to
mathematical multisets, or ``bags'', except that the multiplicity of an element
may be negative. For example, a count operator emits deltas that indicate how
the count for a key has changed; a join may emit an update that installs new
rows in downstream state; and a deletion from a base table generates a
``negative'' update that revokes derived records. Negative updates remove
entries when Noria applies them to state, and retain their negative ``sign''
when combined with other records (e.g., through joins). Negative updates hold
exactly the same values as the positives they revoke and thus follow the same
dataflow paths.

The combined deltas emitted by an operator from the beginning of time
constitutes the operator's current state. This state may not be stored anywhere,
or the delta stream may be \textit{materialized}, in which case the current
multiset of records is stored by the system. It is helpful to think of
\emph{edges} as being materialized, rather than operators or views, since a
materialization is exactly equivalent to the evaluation of the deltas that have
flowed across that edge.

Noria supports \emph{stateless} and \emph{stateful} operators. Stateless
operators, such as filters and projections, need no context to process updates;
stateful operators, such as count, min/max, and top-$k$, maintain state to avoid
inefficient re-computation of aggregate values from scratch. Stateful operators,
like external views, keep one or more indexes to speed up operation. Noria adds
indexes based on \emph{indexing obligations} imposed by operator semantics; for
example, an operator that aggregates votes by user ID requires a user ID index
to process new votes efficiently. Noria's joins are stateless, but require that
the state of their inputs be materialized to allow an update arriving at one
input to join with all relevant state from the other.

\subsection{Example Execution}

\begin{figure}[t]
  \centering
  \includegraphics{diagrams/Example Execution.pdf}
  \caption{Noria's dataflow program to maintain Listing~\ref{f:vote-src}. Text
  describes the update path highlighted in blue. The dataflow inputs are
  considered the ``top'' of the dataflow, and the leaves are at the ``bottom''.
  Parents are ``upstream'' of their ``downstream'' children.}
  \label{f:example-exec}
\end{figure}

Figure~\ref{f:example-exec} shows the dataflow that Noria constructs for
maintaining the Noria program in Listing~\ref{f:vote-src}. At the top are the
entry points into the dataflow\,---\,the operators that represent the schema
base tables\,---\,one for the \texttt{stories} table, and one for the
\texttt{votes} table. Connected to the \texttt{votes} table is a counting
aggregation operator ($\sum$), which corresponds to the internal
\texttt{VoteCount} view. It then feeds into a join operator ($\bowtie$), which
in turn connects to the external \texttt{StoryWithVC} view.

To understand how Noria uses this program to maintain the external view,
consider what happens when a new vote is inserted into \texttt{votes}. The
insertion is introduced into the dataflow as a single-row update with a positive
sign at the \texttt{votes} operator. It stores the update to durable storage,
and then forwards the update as-is to its children; in this case, the count
operator.

The count operator performs a lookup into its own output state for the current
count of the new row's group by column value. The semantics of a count is that
an insertion increments that number by one, so the operator emits a replacement
update for the old state. In particular, the update it produces contains a
negatively-signed record for the old count, and a positively-signed record for
the update count. The replacement also includes the group by column value.

When the join operator receives this replacement update from the count, it
performs a lookup into its other ancestor, \texttt{stories}, for records that
match the join column value of each record of the incoming update. In this case,
both records (the negative and the positive) have the same story identifier as
the join value, so only a single record is returned\,---\,the matching story.
The operator then joins each record in the update with each matching result.
This produces an update that (still) contains one negative and one positive
record, but where each record now includes additional columns from the
\texttt{stories} table.

Ultimately, this two-record update arrives at the operator that represents the
external \texttt{StoryWithVC} view. The changes from the update are applied one
by one, with the negative record removing the entry in the view with the old
vote count, and the positive record adding the replacement entry. Once the
update has been applied, the changes are made visible to readers who will then
start seeing the updated vote count for the story in question.

\section{Consistency Semantics}
\label{s:noria:consistency}

To achieve high parallel processing performance, Noria's dataflow avoids
global progress tracking or coordination. An update injected by a base table
takes time to propagate through the dataflow, and the update may appear in
different views at different times. Noria operators and the contents of its
external views are thus \emph{eventually-consistent}. Eventual consistency is
attractive for performance and scalability, and is sufficient for many web
applications~\cite{eventually-consistent, facebook-memcache, pnuts}.

Noria ensures that if writes quiesce, all external views eventually hold results
that are the same as if the queries had been executed directly against the base
table data. Making this work correctly requires some care. For example, like
most dataflow systems, Noria requires that operators are deterministic functions
over their own state and the inputs from their ancestors. Furthermore, operators
must be distributive over delta addition\footnote{$d_1 + d_2$ produces the union
of all rows in $d_1$ or $d_2$ with the signed multiplicity of each element equal
to the algebraic sum of that element's signed multiplicity in $d_1$ and $d_2$.}
so that evaluating the query using tuple-at-a-time processing is equivalent to
evaluating the whole query at once. Noria operators must also be commutative so
that operators with multiple inputs, like unions and joins, can process their
inputs in any order without coordination\footnote{Maintaining eventual
consistency with partial state requires additional mechanisms, as discussed in
\S\ref{s:correct}.}. The standard relational operators that Noria supports all
have this property.

\paragraph{How Eventual?}
While Noria does not guarantee exactly when a write is visible in a given view,
the time between when a write is acknowledged and when it becomes visible should
be very short in practice. A view is stale only while the write propagates
through the dataflow, so the time before the write manifests depends only on the
height and complexity of the dataflow for the view in question.

\subsection{Prefix Consistency}

Eventual consistency is a woefully vague consistency model\,---\,an eventually
consistent system is allowed to do anything, or nothing at all, as long as reads
\emph{eventually} all return the same value. Most definitions of eventual
consistency also require that that same value reflects the writes issued to the
system, but not all. In practice, eventual consistency is often ``good enough''
despite giving few \emph{guarantees}, and many eventually consistent systems
appear strongly consistent most of the time~\cite{eventual}. Noria in particular
aims to uphold two additional safety properties:

\begin{invariant}
  \label{i:no-spurious}
  All reads reflect each upstream base table change at most once, and no other
  data.
\end{invariant}

\begin{invariant}
  \label{i:prefix-consistency}
  A read that observes all effects of a given base table change also observes
  all earlier changes to that same base table.
\end{invariant}

Invariant~\ref{i:no-spurious} ensures that views include no duplicated
results and no spurious results. Essentially, it guarantees that view results
represent data that is in the base tables, and no other data.

Invariant~\ref{i:prefix-consistency} on the other hand ensures that when the
system makes progress, it does so in a sane manner. If the application issues
two writes $w_1$ and $w_2$ to a base table in that order, then it will not see
$w_2$ reflected but not $w_1$.

Informally, these two invariants taken together establish a sort of per-table
\emph{prefix consistency}. For each base table $b$ that connects to a given view
$v$, there exists some time $t$ such that all changes in $b$ that precede $t$
manifest exactly once when reading from $v$. Effects from changes that succeed
$t$ may manifest fully, partially, or not at all. As Noria continues executing
its dataflow, $t$ increases, and more writes manifest fully.

Note that prefix consistency applies \emph{per base table}. If the application
writes $w_1$ to \texttt{votes} and then $w_2$ to \texttt{stories}, the effects
of $w_2$ may manifest downstream before $w_1$ does.

\begin{inprogress}
  What about sharded base tables?
\end{inprogress}

\subsection{What Can Go Wrong?}

Noria's eventual consistency can lead to reads producing strange results
under certain circumstances in stream processing
systems~\cite{materialize-eventual}. As an example, consider the query in
Listing~\ref{l:always-wrong}. If the maximum value changes frequently enough,
then the outer and inner query may be perpetually ``out-of-sync''. The current
maximum may not yet be present in the outer query, or a new maximum value may
not yet be present in the inner query. The net result of this would be that the
result set of the query would be empty, even though a traditional database would
yield a non-empty query result.

\begin{listing}[h]
  \begin{minted}{sql}
    SELECT data.key FROM data
    WHERE data.value IN
      (SELECT MAX(data.value) FROM data)
  \end{minted}
  \caption{Query that may perpetually produce no results in Noria.}
  \label{l:always-wrong}
\end{listing}

If the max value changes less frequently, and Noria has time to process a new
update through both dataflow paths (the inner and the outer) before the maximum
changes again, Noria will produce the expected non-empty query result. But
nonetheless, the application author probably did not expect that this query
could return an empty result set, however briefly.

Exactly how strange these phenomena get, and how frequently they manifest,
depends on the nature of the queries and the updates. For example, queries that
only access each base table once will only ever produce results that are stale,
but never results that ``look weird''. Queries with self-joins on the other hand
are particularly prone to these temporary inconsistencies. For example, a join
that computes a parent-child relationship between records may briefly reflect a
new record as a child, but not as a parent, or vice-versa.

\begin{listing}[h]
  \begin{minted}{sql}
    SELECT id, state FROM data WHERE state = 1
    UNION
    SELECT id, state FROM data WHERE state = 2
  \end{minted}
  \caption{Query that may briefly produce duplicates in Noria.}
  \label{l:duplicates}
\end{listing}

These inconsistencies ultimately arise due to race conditions in the dataflow
graph. In particular, if two deltas resulting from a single upstream change race
against each another down different dataflow paths that later converge, one
delta will be applied to the final view before the other. This leaves the view
in an intermediate state where a partial effect of the original update can be
observed until the other delta arrives.

Listing~\ref{l:duplicates} gives an example of a query with this kind of race
condition. Imagine that the application changes the \texttt{state} where
\texttt{id = 42} from 1 to 2. In Noria, this is represented by a removal of the
now-outdated record, and an addition of the updated record. The removal follows
the dataflow path for \texttt{WHERE state = 1}, while the addition follows the
dataflow path for \texttt{WHERE state = 2}. One of those updates will arrive
first at the union, and the materialized view. If it is the removal, then reads
will not see a record with \texttt{id = 42} until the addition is processed.
Conversely, if the addition arrives first, reads will see two records with
\texttt{id = 42} (one with \texttt{state = 1} and one with \texttt{state = 2})
until the removal is processed.

To mitigate these kinds of inconsistencies, Noria would either need to enforce
that no reads can happen between the application of one ``half'' and the other,
or somehow hide the partial effects of applying only the first part. The former
requires reads to flow through the dataflow, or at least synchronize closely
with it, which would likely come at a significant penalty to read latency. The
latter would be provided through something akin to multi-version concurrency
control, which allows low-latency reads, but requires timestamps for reads and
writes. Experience from multiple failed designs for logical timestamps in Noria
suggest that producing useful logical timestamps without costly global
coordination is difficult, but not impossible. Implementing this kind of
inconsistency mitigation while retaining Noria's low read latency is left for
future work.

\section{Parallelism}

Servers have many cores, and any high-performance system must be able to take
advantage of multiple cores to take full advantage of the hardware. Noria does
so in several ways. First, it allows reads to happen independently from any
number of threads concurrently. Second, it allows different threads to process
writes in disjoint parts of the dataflow concurrently. And third, it supports
sharding individual operators, or cliques of operators, so that multiple threads
can process disjoint subsets of the data concurrently through the same dataflow
segment. These three mechanisms are described further below.

\subsection{Read Independence}

Since Noria is primarily oriented towards read-heavy workloads, its architecture
is optimized for allowing reads to go ahead at full speed whenever possible. In
particular, Noria does not synchronize reads with reads \emph{or} writes. Unless
a read encounters missing partial state, it never blocks waiting for another
thread.

This is achieved through a concurrency primitive that maintains two instances of
each materialized view, with deduplication between them. Reads go to one view,
and writes to the other. Readers only see updates to the view when a writer
exposes those changes explicitly\,---\,the writer flips an atomic pointer to the
other view, and then waits for all readers to exit the old view before modifying
it again. This can be done on every update, as Noria currently does, or only
occasionally to amortize the cross-core communication penalty. Crucially,
readers do not take locks, and generally operate only on core-local cache lines.

This design allows Noria to use any number of threads to serve reads from any
view. As long as there are cores available, new threads can be added to perform
view lookups and request serialization and deserialization.

\begin{inprogress}
  This design is further discussed in Appendix B?
\end{inprogress}

\subsection{Partitioning}

As part of dataflow planning, Noria divides the dataflow graph into a number of
sub-graphs called \textit{thread domains}. Only a single thread can process
updates in a given thread domain at a time (except with sharding; see below),
and any update that enters a thread domain is processed to completion within
the domain before another update is processed.

State is never shared between thread domains, which means no locks are needed.
Thread domains only ever communicate with one another through messages sent
across the edges of the dataflow, or in the case of upqueries, by sending
messages through dedicated upquery paths set up by the partial subsystem. All
such communication can happen either over the network if the other thread domain
is on a different host, or over an in-memory channel if it is local.

Since thread domains share nothing, some state may be duplicated across
boundaries. For example, a join operator at an incoming edge of a thread domain
must be able to perform lookups into the state of its ancestor, which sits in a
different thread domain. In such a case, Noria will create a thread-local copy
of the join's ancestor's state that it can use locally. Noria's thread domain
assignment heuristics will attempt to draw domain boundaries  such that this
kind of duplication is unnecessary. For example, it will prefer drawing a domain
boundary just before an aggregation (which does not need to look up in the state
of its ancestor), and avoid drawing a domain boundary just before a join.

\subsubsection{Join Consistency}
\label{s:join-state-dupe}

The thread-local copy of lookup state, such as for joins, serves a second
purpose: it mitigates a race condition that would otherwise arise from
cross-domain state lookups. Consider a join operator J with parents L and R. If
R's state was in a different domain than J, then the following can happen:

\begin{enumerate}
  \item R receives and incorporates a delta $d_R$ that adds row $r_R$.
  \item J receives a delta $d_L$ from L that adds row $r_L$.
  \item J performs a lookup into R's state based on $d_L$'s join key. The result
        includes $r_R$, so J emits a delta that adds $r_L \bowtie r_R$.
  \item $d_R$ arrives at J.
  \item J performs a lookup into L's state based on $d_R$'s join key. The result
        necessary includes $r_L$, so J emits a delta that adds $r_L \bowtie r_R$
        a second time.
\end{enumerate}

This issue arises because the lookups bypass deltas that are in-flight between R
and J. The lookups thus get to observe ``the future'', which erroneously causes
J to incorporate new data at two points in time.

Multi-version concurrency control could likely be used to solve this problem,
but would be heavyweight solution that also requires much more significant
synchronization. Especially if J and R happen to be on different physical hosts.
Duplicating R's state across the domain boundary avoids the problem\,---\,since
thread domains process all updates within the domain to completion, there can be
no deltas in flight between R and J, and the lookup will never observe future
state.

\subsection{Sharding}
\label{s:noria:sharding}

To accommodate applications with such a high volume of writes that the
processing at a single operator is a bottleneck, Noria supports sharding an
operator. Multiple threads split the work of handling updates to a sharded
operator, and operate like independent, disjoint parts of the dataflow.

Noria implements static hash partitioning: how an operator is sharded is
determined when it is added to the dataflow, and does not change over the
runtime of the application. Sharding by value ranges and adjusting the sharding
dynamically is left for future work.

Noria shards operators primarily based on how they access state. For example, an
aggregation performs lookups into its own state, and is sharded by the key
column of those lookups. Any other sharding would mean that processing one
update would require coordination among all shards. A join is sharded by the
join key for the same reason. Base tables are sharded by the table's primary
key. Operators that do not perform lookups (e.g., unions) continue the sharding
of their ancestors to avoid unnecessary shuffles.

To shard an operator, Noria introduces two additional nodes in the dataflow: a
\emph{sharder} placed upstream of the sharded operator, and a \emph{shard
merger} downstream of it. The sharder routes incoming updates to the appropriate
shard of the sharded operator, and the shard merger is a union operator that
combines the output of all the shards to a single downstream output stream.
Noria then eliminates unnecessary sharders and shard mergers, such as if an
operator and its ancestor are sharded the same way.

Sharding boundaries are naturally also thread boundaries, though two connected
thread domains may also be sharded differently. Or, phrased differently, Noria
may partition a chain of operators that are all sharded by the same column into
multiple thread domains to increase parallelism.

\begin{comment}
  Sharding is tracked based on "ultimately source column". (column tracing)
\end{comment}


\chapter{Partial State}

Noria, as described in \S\ref{s:noria}, uses significant amounts of memory. All
results for all queries must be materialized, and unlike traditional caching
approaches, unimportant cached results cannot be evicted to free up memory. To
address the high memory use of traditional materialized views, this thesis
proposes \textit{partially materialized state}. Partial state enables Noria to
store and maintain only a subset of a materialized view's contents, and to
compute missing state on demand. Partial state also enables Noria to implement
eviction, so that the materialization cost is kept low even as the underlying
workload changes.

This chapter discusses the partially stateful model and its components. The next
chapter examines the practical challenges that arise when partial state is
implemented in a dataflow system.

\section{Missing State}

At the core of the partial model is the notion of \textit{missing state}.
Missing state indicates that a particular value is not yet known, and must be
computed on demand if the application queries for it. State can be marked as
missing both in state that is internal to the dataflow, like the state of an
aggregation, and in externally visible state like Noria's query result caches.

In Noria, most state starts out as missing, and is populated driven by what data
the application queries for. This also allows Noria to quickly adopt new views,
since in the common case no computation need happen when additional operators
are added.

Missing state manifests as missing entries in indices in the system. For
example, an aggregation that groups by the author of a given article would have
an index over the author column. An entry can be missing in that index to
signify that the aggregated value for a given author is unknown. If state
carries multiple indices, a particular base table row may be present in one
index, but missing in another. Indexes over a given state are either all partial
or none of them are.

If, while processing an update, the system encounters missing state, this
indicates that the update does not affect query results that the application has
indicated interest in. In such a case, the system is presented with two options:
eagerly compute the missing state before proceeding, or discard the update. To
avoid unnecessarily maintaining unimportant cached results, updates are dropped
in this case.

An important corollary of the above is that partial state must be enabled
on all stored state \emph{below} any partial state. More formally:

\begin{invariant}
  \label{i:partial-above-full}
  It is illegal for the dataflow to contain state for two nodes $A$ and $B$
  where $A$ is an ancestor of $B$, $A$ uses partial state, and $B$ does not use
  partial state.
\end{invariant}

To see why, consider what would happen if an update arrives at $A$ for a missing
entry. $A$ would discard that update, and $B$'s state would remain perpetually
stale.

\section{Upqueries}

If an application requests data that is found to be missing, the system issues
an \textit{upquery} to compute the requested data. Upqueries flow ``up'' the
dataflow graph, towards the base tables at the ``top'', and constitute a request
for the target of the upquery to replay past data. Upqueries may recurse if the
requested state is not available at the initial target.

The response to an upquery takes the form of a regular update that flows through
the normal dataflow program. This update is a snapshot that combines all past
deltas pertinent to the upquery into a single update. That single update holds
only positive deltas that represent the current set of relevant records.

Operators are generally not aware whether they are processing an update that
resulted from a base table change or an upquery. The upquery response flows
in-line with other dataflow updates, and follows the edges of the dataflow.

Upquery responses are special in two key ways. First, they only propagate along
edges towards the operator that issued the upquery, so that one upquery does not
populate the relevant data in the state of \emph{every} operator. And second,
if an operator encounters missing state while processing an upquery response
update, it does not discard that update, but instead does the work necessary to
process that update fully. This process is described in further detail below.

When an application query encounters missing state in a view, the system needs
to know what upqueries to issue. The set of necessary upqueries for each view is
that view's \textit{upquery plan}. The system determines upquery plans by
analyzing each view's query when that view is created, and deciding how best
to recompute its results. It does so by finding all \emph{possible} upquery
plans, choosing among them, and then informing all involved domains of the
chosen plan.

\subsection{Key Provenance Tracing}

To determine what upqueries can be issued to reconstitute missing entries in a
given index, the system must trace the view's parameter column back to a column
in upstream state. The intuition here is that in order to answer the
application's query of ``give me the results where column $C$ has value $x$'',
the system must be able to replay rows where $C = x$ from somewhere. Or, phrased
differently, when the output for $C = x$ is missing, the system must have a way
to get at the inputs that \emph{generate} $C = x$. As an example, if a view
counts books by a given author, and the current count for author $a$ is missing,
the system must be able to somehow produce all books by author $a$.

\begin{figure}[t]
  \centering
  \includegraphics{diagrams/Key Provenance.pdf}
  \caption{Key provenance for each column in the \texttt{StoriesWithVC} view
  from Listing~\ref{f:vote-src} and Figure~\ref{f:example-exec}. Notice that
  \texttt{story\_id} has multiple base table origins, and \texttt{vcount} does
  not trace back to any base table columns. The query only uses
  \texttt{story\_id} as a parameter, so only its provenance is used to choose
  the upquery path.}
  \label{f:key-prov}
\end{figure}

More generally, in order to recompute the results where $C = x$ in some view
$V$, the system must determine the \textit{key provenance} of $C$; where $C$
``came from''. The system computes key provenance by tracing columns ``up'' the
dataflow to where they originate, which results in a provenance graph like the
one shown in Figure~\ref{f:key-prov} for the \texttt{StoriesWithVC} view from
Listing~\ref{f:vote-src}. The figure illustrates two important properties of
key tracing:

\begin{enumerate}
  \item An output column may trace to multiple input columns if it corresponds
    to the join column in a join, or if it passes through a union. The
    \texttt{story\_id} column, for example, originates both in
    \texttt{stories.id} and \texttt{votes.story\_id}.
  \item An output column may be entirely computed, and thus have no association
    with a column in the inputs. The \texttt{vcount} column is computed by the
    \texttt{VoteCount} aggregation, and does not exist in the input data set.
\end{enumerate}

In Listing~\ref{f:vote-src}, the system is asked to parameterize
\texttt{StoriesWithVC} by the \texttt{story\_id} column. The key provenance
graph tells the system that it can request input data for a given
\texttt{story\_id} by sending an upquery either to the \texttt{stories} table
using the \texttt{id} column, or to the \texttt{vote} table using the
\texttt{story\_id} column. Since there exists a way to replay the input data for
a missing output entry in this case, the \texttt{StoriesWithVC} view can be made
partial.

\paragraph{Missing Provenance.}
Consider what would happen if Listing~\ref{f:vote-src} had \texttt{WHERE vcount
= ?} as its parameter instead. If an application query misses in that case, the
upquery would have to be sent to \texttt{VoteCount}, and query for ``all stories
whose vote count is $x$''. If that state is present, all is well, but if
\texttt{VoteCount} is itself partial, and is missing the state for
\texttt{vcount = x}, there's a problem. The system now has no way to compute the
missing state except by replaying \emph{all} state in \texttt{vote} without an
index. This would be equivalent to a full table scan in a traditional database.
The system's only%
%
\footnote{The system cannot disable partial state just for
\texttt{StoriesWithVC}, since that would violate
Invariant~\ref{i:partial-above-full}.}
%
efficient option is to disable partial
state for \texttt{VoteCount}. This ensures that any upquery to it never misses,
and therefore a table scan is never needed.

\paragraph{Asymmetric Provenance.}
The join in Listing~\ref{f:vote-src} uses a inner join ($\bowtie$), and the
system is therefore free to upquery \emph{either} side. If it upqueries the
``left'' side of the join, the regular processing pipeline will perform the
necessary lookups into the ``right'' side of the join, and vice-versa. However,
if the view query used a left or right \emph{outer} join, the system must
upquery a particular side of the join. For a left join, it must upquery the left
ancestor, or risk missing rows in the left ancestor that have no matching rows
in the right ancestor. For a right join, the same logic applies, but mirrored to
the right ancestor.

\paragraph{Disjoint Provenance.}
If the provenance of a column crosses a union, \emph{all} ancestors of that
union must be upqueried, as opposed to just one as is the case with upqueries
through a join. Unlike with a join, the regular dataflow processing of the
upquery response through a union will not also bring along results from the
other ancestors, so the requesting operator must ask them all individually.

\subsection{Path Selection}

Once the system has obtain a set of candidate upquery paths through key
provenance, it must decide on an upquery plan based on those paths. If there is
only one candidate, the choice is trivial. But with symmetric joins, multiple
candidate paths may be generated. Here, the system is free to use whatever
heuristics it sees fit to decide which side of a symmetric join to send
upqueries to. For example, it may choose to send upqueries to the larger of the
joined inputs so that fewer lookups are necessary when processing the response.

In addition, key provenance tracing produces upquery paths that reach all the
way back to the origin of a column, which is usually located at the base tables.
However, it would be inefficient for operators to issue upqueries all the way to
the base tables on every miss. Some intermediate state may already have the
necessary data, and the upquery data could be sourced from there instead. The
system therefore trims the paths from key provenance such that only the suffix
of operators starting at the last materialized state are included. If an upquery
reaches its origin and finds that the requested state is missing there too, a
second upquery is issued using the origin's upquery paths, and only when that
upquery resolves does the original upquery resume. Upqueries may recurse all the
way up to the base tables this way, but avoid doing so if any intermediate state
can be re-used.

This process leaves the system with a set of paths that the system should
upquery when it encounters a missing entry in a given state.

\subsection{Implementing a Plan}

Once the system has decided on a plan, that plan is communicated to all domains
that appear along each path in the plan. This is necessary so that each domain
knows where to route upquery responses that are part of a given plan, and does
not disseminate the response to the entire downstream dataflow.

When the system announces the upquery plan, it may also add additional indices
to existing state to facilitate efficient execution of the new upqueries. In
particular, when an upquery arrives at the materialization it wants to source
data from, the system needs an efficient way to find the requested data.
Concretely, the system needs an index on the materialization whose key matches
the lookup key of the upquery. The key provenance information from
Figure~\ref{f:key-prov} gives the system all the information it needs to set up
these indexes: an index is needed on the upquery key column on each state on the
chosen upquery paths. In the case of the view from Listing~\ref{f:vote-src}, an
index is needed on \texttt{StoriesWithVC.story\_id}, as well as either
\texttt{stories.id} or both \texttt{VoteCount.story\_id} and
\texttt{votes.story\_id}\footnote{An index is needed on \texttt{votes.story\_id}
since the upquery to \texttt{VoteCount} may recurse.}, depending on which
upquery path the system chooses across the join.

\section{Eviction}

Over time, the subset of data that the application cares about tends to change.
When it does, query results that were accessed previously may no longer be
important to maintain as they are no longer accessed. Partial state allows the
system to cater to such changing application patterns by \textit{evicting} state
entries after they have been computed. When an entry is evicted, it is marked as
missing, and subsequent requests for that state trigger an upquery as usual for
missing state.

When the system evicts an entry from state in the middle of the dataflow, a
situation similar to what inspired Invariant~\ref{i:partial-above-full} arises.
The missing entry may cause an update to be discarded even though downstream
entries must see that update to avoid. For example, consider what happens if an
operator counts the number of votes per author, and contains a count
of 7 for the author ``Jane''. Then, the state for ``Jane'' is evicted from some
operator upstream of the count. If an update now arrives at the operator where
``Jane'' is missing; it would discard the update, and the downstream count would
remain perpetually stale.

To avoid such permanently stale entries, the system must ensure that when an
entry is evicted, any dependent state downstream is also evicted. More formally,
it must ensure that:

\begin{invariant}
  \label{i:missing-suffix}
  If an operator encounters a missing state entry while processing a record $r$
  in an update, any downstream state that reflects $r$ must be evicted.
\end{invariant}

Upqueries as described thus far, without evictions, generally uphold this
invariant\footnote{Though not always; see \S\ref{join-evictions}}. As upqueries
recurse up the dataflow, missing state is filled in ``from the top down'' until
the operator that originally issued the upquery is presented with the requested
state. Any misses that occur during the processing of an upquery causes
recursive upqueries, which fill that missing state before proceeding.

To uphold the property in the face of evictions, the system issues evictions
downstream whenever it decides to evict entries from state in the middle of the
graph. This ensures that any future update that touches the evicted state can
safely be discarded, as any relevant downstream state has been discarded as
well.

Invariant~\ref{i:missing-suffix} implies
Invariant~\ref{i:partial-above-full}\,---\,since state cannot be evicted from
non-partial state, the system would be unable to satisfy
Invariant~\ref{i:missing-suffix} should it ever encounter missing state upstream
of the non-partial state.


\chapter{Maintaining Correctness}
\label{s:correct}

The addition of partial state significantly changes how query results are
computed by the underlying dataflow system. This chapter demonstrates how
partial state is implemented in Noria such that it produces results similar to
those Noria would produce without partial state.

\section{Defining Correctness}

In order to discuss the degree to which partial achieves this goal, it is
necessary to first define what constitutes ``correct'' behavior, both in Noria
without partial, and with the introduction of partial.

\resume

\subsection{Correctness in Noria}

\subsection{Correctness with Partial State}

To give some intuition for why this problem is challenging, we first need to
understand what the goal of the system as a whole is. Ultimately, the partial
invariants all serve to maintain one principal property:

\begin{quote}
	If data stops flowing into the dataflow, the dataflow will eventually
	quiesce. When it does, for every key in every state, the value for that
	key is either missing, or it reflects the effects of each input to the
	system applied exactly once. A subsequent query for any missing key in
	any materialization populates the state for the missing key consistent
	with the property above for non-missing state.
\end{quote}

The intuition here is that Noria must \emph{at least} eventually do the right
thing. That is, it must make sure that all the data the application inserts into
the dataflow is considered, that none of it is double-counted, and that no other
spurious data is added. Unless, of course, the application has inserted dataflow
operators that double-count, in which case they should be exactly
double-counted.

We want Noria to provide stronger guarantees than eventual consistency whenever
possible and, in the common case, it does. Specifically, for most queries, Noria
ensures that a read from any given view sees complete query results as of some
recent time at each dataflow input. That is, for a given view, for each input
that feeds into that view, the view reflects a prefix of the data ingested by
that input. I call this \emph{prefix consistency}. Each view is also
continuously kept up to date; any new input is reflected in the view shortly
after being ingested, subject only to the propagation delay in the dataflow.

Noria does not necessarily provide prefix consistency when there are
\textbf{multiple} paths from a given dataflow input to a given view, such as
through a self-join. Depending on the precise semantics of the paths, this can
cause a view to briefly reflect \textbf{some} of the effects of newly inserted
data, but not all. For example, consider a self-join that computes a
parent-child relationship between records. If the application removes a record
$A$, that dataflow input must be processed along two edges. When it has been
processed by one edge, but no the other, the downstream view will briefly
continue to include $A$ as a child, even though it no longer appears as a
parent. This inconsistency is rectified once the dataflow input is also
processed on the second edge.

This problem is not directly related to partial state\,---\,Noria exhibits this
behavior when all state is fully materialized. However, partial state must work
in the context of such temporary inconsistencies. Furthermore, partial state
should not exaggerate these problems by introducing additional inconsistencies.

There are several situations that arise in a real dataflow implementation that
make even this seemingly simple property difficult to uphold. I sketch the
primary ones below, and give brief descriptions of my proposed solutions. In my
thesis, I will go into these in greater detail. I will also provide a more
comprehensive analysis of the possible inconsistencies that can arise if these
situations are not handled correctly by the partial state logic.

At its core, partial state introduces two new conditions into the dataflow that
were not previously present. First, multiple updates may now be collapsed and
re-processed through the dataflow as a single, consolidated update in response
to an upquery. These consolidated updates represent \textit{snapshots} of
upstream state, and the system must ensure that these snapshots do not introduce
duplicate or spurious query results, or fail to include relevant data.

Second, with partial state in place, \emph{any} operator processing may
encounter missing state. When that happens, the system

How do we know that Noria is eventually consistent?

First, how do we know that Noria without partial state is eventually
consistent. And second, how do we know that Noria remains eventually
consistent with partial state.

We don't have a formal proof for either. Ignoring implementation bugs, the
informal argument goes something like:

For partial state, we need to show that each of the things that have been
changed preserve eventual consistency. First, sending a snapshot of past state
(upquery response) is equivalent to sending the individual updates that made
up that snapshot since the operators are distributive and commutative. Second,
upquery results can race with related updates, which is handled by the various
challenges I listed in the proposal:

 - sending them inline in the dataflow (for single-branch)
 - union buffering (for unioned branched dataflow)
 - special join upquery handling (for joined branched dataflow)

And third, partial state means that lookups can miss internally in the
dataflow, which is handled by join eviction when related state may still be
present downstream, and by discarding the update when it is not.

\section{Linear Dataflow}

First, consider a single strand of dataflow, where each operator has at most one
input and at most one output. For partial state to be correct, it must be the
case that computing missing results with an upquery that combines all past
deltas into a single update produces the same results as processing the deltas
one-at-a-time.

\begin{inprogress}
  What about discarded updates in the paragraph below?
\end{inprogress}

The deltas that flow through the dataflow system represent changes to the
logical output state of the operator that produced the delta. If a base table
produces a negative delta for a row $r$, it means that row $r$ is no longer part
of that base table's current state. An upquery fetches current state, which is
the sum of all past deltas emitted by the queried operator, and then feeds it
through the same chain of dataflow operators that individual deltas go through.
For upquery processing to be equivalent to one-at-a-time delta processing, it
must be the case that with operators $f_1$ through $f_N$, and past deltas $d_1$
through $d_M$:

\begin{eqnarray*}
  \sum^M_{i=1}\left(f_N \circ \dots \circ f_1\right)\left(d_i\right) = \
  \left(f_N \circ \dots \circ f_1\right)\left(\sum^M_{i=1}d_i\right)
\end{eqnarray*}

With a single operator, this trivially holds since all Noria operators are
distributive over delta addition:

\begin{eqnarray*}
  \sum^M_{i=1}f\left(d_i\right) = \
  f\left(\sum^M_{i=1}d_i\right)
\end{eqnarray*}

Using this property, and the fact that all operators produce and consume deltas,
it is possible to ``shift'' the delta sum across operator compositions:

\begin{eqnarray*}
  \sum^M_{i=1}\left(f_{n+1} \circ f_n\right)\left(d_i\right) &=& \sum^M_{i=1}f_{n+1}\left(f_n\left(d_i\right)\right) \\
  &=& f_{n+1}\left(\sum^M_{i=1}f_n\left(d_i\right)\right) \\
  &=& f_{n+1}\left(f_n\left(\sum^M_{i=1}d_i\right)\right) \\
  &=& \left(f_{n+1} \circ f_n\right)\left(\sum^M_{i=1}d_i\right)
\end{eqnarray*}

Therefore, the same ultimate state results whether the system executes each
dataflow operator in sequence on individual deltas, or whether it first sums
all the deltas into a single update, and then executes the operators in sequence
over that. Or, stated differently, if normal dataflow processing produces the
correct result, so too must processing a combined upquery response.

\section{Diverging Dataflow}

Dataflow graphs in real applications are rarely linear. They contain branches
where the dataflow diverges, such as if two views both contain data from the
same table. When the dataflow diverges, the biggest change from partial state is
that upstream operators may receive multiple upqueries for the same data. This
happens if multiple downstream views encounter missing entries that rely on the
same upstream data.

The primary concern in this case is that the multiple upquery responses not
result in data duplication\,---\,if a stateful operator processes two upquery
responses that both reflect some base table row, the effects of that row are now
duplicated in the operator's state. Since upquery results only ever flow along
the same edges that the upquery followed on its way up the dataflow
(\S\ref{s:upqueries}), duplicates are only a concern for state that is on the
upquery path\,---\,state on other branches will never see the upquery response
in the first place.

\S\ref{s:upquery:selection} noted that upquery paths are trimmed such that they
only reach back to the \emph{nearest} materialized state to the target. Beyond
improving efficiency, this is also important for correctness. It ensures that
there are no stateful operators on the upquery path between the source and the
destination. If there were, that operator's state would be used as the upquery
source instead. Since it is safe to process the same record through a stateless
operator multiple times, this ensures that the processing of the upquery
response on the path to the target state never duplicates effects.

\section{Merging Dataflow}

In practice, most applications include at least one join or union in their
queries. When they do, strands of dataflow combine to produce joint output that
depends on multiple inputs. And crucially, such dataflow constructions introduce
the possibility of data races. Now, updates may arrive to an operator from two
inputs in parallel, and the operator may process either update before the other.
Furthermore, upqueries must now retrieve data from \emph{all} ancestors, and
ensure that they are combined in such a way that no duplicate or spurious data
is introduced, and no data is missed.

How upqueries work across multi-ancestor operators depends on the semantics of
that operator. The two primary relational multi-ancestor operators, unions and
joins, are discussed below.

\subsection{Unions}

Unions merely combine the input streams of their ancestors, and includes little
processing beyond column selection. An operator that wishes to upquery past this
operator must therefore \emph{split} its upquery; it must query each ancestor of
the operator, and take the union of the responses to populate all the missing
state.

With concurrent processing, the multiple resulting responses may be arbitrarily
delayed between the different upquery paths, which can cause issues. Consider
a union, U, across two inputs, A and B, with a single materialized and partial
downstream operator C. C discovers that it needs the state for $k = 1$, and
sends an upquery for $k = 1$ to both A and B. A responds first, and C receives
that response.

First, C needs to remember that the missing state is still missing, so that it
does not expose incomplete state downstream. For example, if it received an
upquery for $k = 1$, it could not reply with \textbf{just} A's state.

Now imagine that both A and B send one normal dataflow message each, and that
they both include data for $k = 1$. When these messages reach C, C faces a
dilemma. It cannot drop the messages, since the message from A includes data
that was not included in A's upquery response. If it dropped them, that data
would disappear forever, and results downstream would be permanently stale. But
it also cannot apply the messages, since B's message includes data that will be
included in B's eventual upquery response. If it did, that data would be
duplicated.

\begin{listing}[h]
  \begin{minted}{python}
if is_upquery_response(d):
  buffered <- buffer[upquery_path_group(d)][key(d)]
  if len(buffered) == ninputs - 1:
    # this is the last upquery response piece.
    # emit a single, combined response
    emit(sum(buffered) + d)
    delete buffered
  else:
    # need responses from other parallel upqueries.
    buffered[from(d)] = d
    discard(d)
else:
  # this is a normal dataflow delta.
  # see if any changes in the delta
  # affect buffered upquery responses.
  for group_id, key_buffers in buffer:
    for change in d:
      change_key <- change[key_column(group_id)]
      # note the dependence on from(d) below.
      # changes from parents that have not produced
      # an upquery response yet are ignored; they
      # are represented in the eventual response.
      buffered <- key_buffers[change_key][from(d)]
      if buffered:
        buffered += change
  # always emit the delta, as other downstream
  # state may depend on it. any operator that is
  # waiting for missing state will discard.
  emit(d)
  \end{minted}
  \caption{Pseudocode for union buffering algorithm upon receiving a delta
  \texttt{d}. \texttt{buffer} starts out as an empty dictionary.
  \texttt{upquery\_path\_group} is discussed in the text.}
  \label{l:union-buffer}
\end{listing}

To mitigate this problem, unions must \textit{buffer} upquery results until
\emph{all} their inputs have responded. In the meantime, they must \emph{also}
buffer updates for the buffered upquery keys to ensure that a single, complete,
upquery response is ultimately emitted. Listing~\ref{l:union-buffer} shows
pseudocode for the buffering algorithm.

For unions to buffer correctly, they must know which upquery responses belong to
the ``same'' upquery. If there is only one upquery path through the union to
each ancestor, this is straightforward, as all upquery responses for a key $k$
are responses to the same upquery, and should be combined. However, in more
complex dataflow layouts, this is not always the case.

\begin{figure}[t]
  \centering
  \includegraphics{diagrams/Chained Unions.pdf}
  \caption{Chained unions. Only nodes $a$, $b$, and $v$ hold state. Highlighted
  in orange are two upquery paths that $\cup_1$ must combine upquery responses
  for.}
  \label{f:chained-union}
\end{figure}

Figure~\ref{f:chained-union} shows a dataflow segment where the precise grouping
mechanism is important (\texttt{upquery\_path\_group} in the code listing).
There are three unions in a chain, which makes eight distinct upquery paths. If
$v$ encounters missing state, it must therefore issue eight upqueries, one for
each path. $a$ and $b$ both appear as the root of four paths, and will be
upqueried that many times. The issue arises at the unions, which need to do
the aforementioned union buffering.

Ultimately, a single upquery response must reach $v$. This means that $\cup_3$
must receive two upquery responses, one from $e$ and one from $f$, which it must
then combine. So $\cup_2$ must \emph{produce} two upquery responses, one
destined for $e$ and one for $f$. This in turn means that $\cup_2$ must receive
two upquery responses from $c$, and two from $d$. Which again means that
$\cup_1$ must produce four responses, two for $c$ and two for $d$, out of the
eight responses it receives (four from $a$ and four from $b$).

These are all the upqueries that pass through $\cup_1$:
        
\begin{multicols}{4}
\begin{description}
  \item [1.] $a\,\to\,c\,\to\,e$
  \item [2.] $a\,\to\,c\,\to\,f$
  \item [3.] $a\,\to\,d\,\to\,e$
  \item [4.] $a\,\to\,d\,\to\,f$
  \item [5.] $b\,\to\,c\,\to\,e$
  \item [6.] $b\,\to\,c\,\to\,f$
  \item [7.] $b\,\to\,d\,\to\,e$
  \item [8.] $b\,\to\,d\,\to\,f$
\end{description}
\end{multicols}

$\cup_1$ must combine these in pairs such that every downstream union receives
the responses that they expect from each of their inputs. The grouping that
achieves this is:

\begin{multicols}{2}
\begin{description}
  \item [1/5.] $a/b\,\to\,c\,\to\,e$
  \item [2/6.] $a/b\,\to\,c\,\to\,f$
  \item [3/7.] $a/b\,\to\,d\,\to\,e$
  \item [4/8.] $a/b\,\to\,d\,\to\,f$
\end{description}
\end{multicols}

The key observation is that the distinction between $a$ and $b$ does not matter
any longer downstream of $\cup_1$; a delta that arrived from $a$ is
indistinguishable from one that arrived from $b$. Similarly, the distinction
between $c$ and $d$ no longer matters past $\cup_2$, and the same for $e$ and
$f$ past $\cup_3$. \texttt{upquery\_path\_group} is thus defined as a unique
identifier for $v$'s upquery plan plus the sequence of nodes between the union
and the target of the upquery response.

\subsection{Joins}

For unions, as we saw above, the upquery must go to all the ancestors. For joins
on the other hand, the upquery must only go to \textbf{one} ancestor. This is
because when a join processes a message from one ancestor, it already queries
the ``other'' ancestor and thus pulls in any relevant state. If both sides of
the join were queried, the processing of the upquery responses at the join would
produce duplicates of every record.

For a symmetric join, either ancestor can be the target of the upquery, whereas
for an asymmetric join (like a left join), the upquery \emph{must} go to the
``full'' side\,---\,the side from which all rows are yielded. Otherwise, the
upquery may produce only a subset of the results for the join.

\subsubsection{Dependent Upqueries}

When a upquery response passes through a join operator, the join performs
lookups into the state of the other side of the join. With partial state, those
lookups may themselves encounter missing entries. When this happens, a problem
arises: the system \emph{must} produce an downstream upquery response because
the application is waiting for it, but it cannot produce that response since
required state is missing.

For the purposes of exposition, and without loss of generality, the text below
refers to the join ancestor that was upqueried as the left side, and the
ancestor that a lookup missed in as the right side.

The join must issue an upquery to the right hand side for the state that is
missing to complete the processing of the original upquery response. However,
this \textit{dependent upquery} may take some time to complete, and the system
must decide what to do in the meantime. Recall that the join is still in the
middle of processing an upquery response.

An obvious, but flawed strategy is to have the join block until the response
arrives. This would not only stall processing of deltas from the left parent,
but also leads to a deadlock. In order to observe the eventual upquery response,
the join's domain must continue to process incoming messages to the right parent
(\S\ref{s:join-state-dupe}). But in doing so, it may encounter a different
upquery response from the right parent. That upquery response may require a
lookup into the left parent's state, which may itself encounter missing entries.
The join is then forced to block on both inputs perpetually.

Instead, the join \emph{discards} the current upquery response, and notes down
the upquery parameters that triggered it, and the missing state it is waiting to
be filled. It then continues processing as normal. When all the missing entries
are eventually filled in, the system \emph{re-issues} the original upquery to
the join's left parent using the parameters it saved. This time, when the
upquery response arrives, the lookups into the right-hand parent's state, all
required entries should be present, and the downstream upquery response is
produced. As far as the downstream dataflow is concerned, nothing abnormal has
happened\,---\,the upquery response just took longer to arrive.

\subsubsection{Incongruent Joins}

The system must guarantee that all data relevant to a given state entry
eventually reaches the operator that holds that entry. A corollary of this is
that the system cannot discard messages that may affect non-missing, downstream
state. Normally, this is the case, since upqueries traverse the dataflow from
the leaves and up, and fill entries from the top down as the responses flow down
the dataflow. If some key $k$ is present in a materialization $m$, it must also
be present at every materialization above $m$ from the upquery chain that
ultimately produced the entry for $k$ in $m$.

Unfortunately, certain query graphs produce dataflow where more than one key is
used in the dataflow to compute an entry. Consider a dataflow that joins two
inputs, \texttt{story} and \texttt{user}, on the story's author field. A
downstream operator issues an upquery for story number 7. The upquery is issued
to \texttt{story}, which produces a message that contains story number 7 with,
say, author ``Elena''. That message arrives at the join, which issues a
dependent upquery to \texttt{user} for ``Elena''. When that dependent upquery
resolves, the join produces the final upquery response, and the state for story
number 7 is populated in the downstream materialization.

Next, an editor changes the author for story number 7 to ``Talia''. This
takes the form of a delta with a negative multiplicity record for \texttt{[7,
"Elena"]} and a positive one for \texttt{[7, "Talia"]}. When this delta arrives
at the join, it may now miss when performing the lookup for ``Talia''. According
to the partial model so far, the join should now drop \texttt{[7, "Talia"]}, and
only allow the negative for ``Elena'' to propagate to the downstream
materialization. When that happens,  the state for article number 7 becomes
empty (though not missing), and any subsequent read for article number 7
receives an empty response, which violates correctness.

What happened here was that the entry for key $k$ in the leaf-most
materialization depends not only on state entries indexed by the same $k$
upstream, but also on state entries indexed by other keys upstream. While $k$
must be present upstream, no such guarantee exists for other keys.

This is a result of \textit{incongruent joins}; joins whose join column is not
the same as the downstream key column. Incongruence is determined with respect
to each upquery path that flow through a join. In the case above, the author
join is incongruent with an upquery on the story number column, since the join
column is the author column. However, the join is congruent with upqueries from
a hypothetical downstream view that is keyed by author instead. A join that is
incongruent with any upquery path that flows through it is considered an
incongruent join.

The system can easily recognize incongruent joins through key provenance
analysis\,---\,if an upquery flows through a join, and the upquery column is not
the same as the join column, the join is incongruent. If the system then
observes that an incongruent join encounters missing state while processing a
delta at runtime, it must take action to ensure that downstream state remains
correct. Since the domain that processes the join cannot produce a valid delta,
and does not know what state is present and missing in the downstream dataflow,
its only option is to issue an eviction for any downstream state that \emph{may}
be rendered stale. Concretely, if an incongruent join processes a record $r$ and
encounters a missing state entry, it should issue an eviction downstream on all
incongruent upquery paths using the appropriate values from $r$. For example, if
the join column is $c_j$, and upquery path $u_i$ through the join is keyed by
column $c_i$, then the join should issue downstream evictions of $r[c_i]$ for
each $u_i$ where $c_i \neq c_j$.

\section{Sharding}

\resume

Noria supports sharding cliques of operators to increase the throughput
of particular sections of the dataflow. Shards of an operator execute in
parallel, without synchronization. Edges that cross from an unsharded
operator to a sharded one split its outgoing updates using hash
partitioning. Edges that cross back have an implicit union injected to
merge the sharded results. Edges that cross from one sharding to a
different sharding are merged and then split again. Upqueries must also
work when Noria decides to shard operators in this way.

Upqueries across a sharding boundary are a complicated affair. The
operator that issues the upquery must determine which shard or shards to
send the upquery to. If it queries multiple shards, the responses from
those shards are subject to the same multi-ancestor issue as unions.
When a response to the upquery comes back, it must be specifically
routed to only the requesting shard, so that it does not accidentally
populate the state of other shards. This logic must work even if
multiple shards issue an upquery for the same key concurrently. Or,
worse yet, if a single upquery must traverse \textbf{multiple} sharding
boundaries.

\paragraph{Noria solution}
Key provenance informs operators whether an upquery for a given column should be
sent to all shards, or just one shard, of the upquery source. This information,
as well as the shard identifier of the requesting operator, is included in the
upquery itself, and in the eventual response. Sharding unions buffer upquery
responses that originated from more than one shard (like regular unions). Shard
``splitters'' ensure that responses only arrive at the requesting shard using
the requestor information in the response.

\begin{inprogress}
  Diamonds?
\end{inprogress}


\chapter{Evaluation}

This thesis is built on the belief that view materialization is useful, but
prohibitively costly to use with current solutions. It presents partial state as
a solution to this problem that allows retaining the benefits of view
materialization at a fraction of the cost. Section \ref{s:eval:why} evaluates
the validity of this assumption, and the efficacy of partial state as a
solution.

Existing application deployments tend to implement their own ad-hoc caching
logic, usually by placing a dedicated cache in front of the database. Some
larger companies have also built comprehensive tooling to manage their caches
correctly. Section \ref{s:eval:alts} looks at how Noria serves as an attractive
alternative to these approaches.

With partial state, only a subset of each view is materialized, and missing
results are computed on-demand. Partial state thus presents a trade-off between
memory use (cache size) and tail latency (miss rate). Section \ref{s:eval:cost}
explores the effects of this trade-off.

Developers may be hesitant to switch applications that already use a cache today
to Noria without some evidence that their performance won't regress.
Unfortunately, the impact Noria would have on any given application is highly
dependent on the particulars of each application, and is thus hard to measure.
To demonstrate that Noria can offer comparable absolute performance to existing
caching solutions, section \ref{s:eval:kvperf} compares Noria lookup performance
to that of Redis under ideal caching conditions.

An added benefit of partial state is that it allows applications to introduce
new queries/views without materializing it entirely up-front. This enables fast
adoption of new queries, but also means that queries to new views are initially
slow. Section \ref{s:eval:mig} evaluates these partial state migrations compared
to traditional all-at-once materialization.

The ability of partial state to reduce the memory use of view materialization
depends on skew in the application's data and access patterns. It allows Noria
to reduce latency for a significant fraction of requests by keeping only a few
results cached. While it is impossible to predict the skew for an arbitrary
application's data and queries, section \ref{s:eval:patterns} gives a simplified
theoretical model to help with estimation.

\section{Experimental Setup}
\label{s:eval:setup}

The experiments in this chapter primarily use the Lobsters news
aggregator web application at \url{https://lobste.rs}~\cite{lobsters}. This
application was chosen because it is open-source (so we can see what queries it
issues), because it resembles many larger-scale applications (like Hacker News
or Reddit), and because statistics about the site's data and access patterns are
available~\cite{lobsters-data}.

The evaluation uses a workload generator that issues page requests
according to the available statistics~\cite{generator}. It does not run the
real Lobsters Ruby-on-Rails application, as it is prohibitively slow. Instead,
all experiments use an adapter that turns page requests directly into the
queries the real Lobsters code would issue for that same page request. The
generator supports scaling up the rate of access and user count to emulate a
larger user base for benchmarking.

\begin{figure}
  \begin{tabular}{ p{0.8in} | r | r | r | p{2.9in} }
    Page & \% & W & Q & Description \\
    \hline
    Story & 55.8 & 1 & 14 & Renders an individual story's page, including its
    popularity score, comments, and the scores of its comments.\\
    Frontpage & 30.1 & 0 & 14 & Lists the 25 most highly scored stories, along
    with their authors and scores.\\
    User & 6.7 & 0 & 7 & Renders a user summary page, including what story
    ``tags'' they contribute to.\\
    Comments & 4.7 & 0 & 9 & Like the frontpage, but for comments.\\
    Recent & 1.0 & 0 & 14 & 25 most recently added stories, along with their
    authors and scores.\\
    Vote & 1.2 & 1 & 2 & Vote up/down a given comment or story.\\
    Comment & 0.4 & 2 & 5 & Add a new comment to a story.\\
  \end{tabular}
  \caption{Pages in Lobsters. \% indicates the percentage of requests that load
  the given page. W is the number of writes loading the page requires, and Q is
  the number of queries it issues.}
  \label{t:lobsters-pages}
\end{figure}

The various pages in Lobsters differ significantly in what queries they issue,
how many queries they issue, and the extent to which they are read or write
heavy. Table~\ref{t:lobsters-pages} gives an overview of the frequency of
requests for each page, what loading the page entails, and a brief description
of that page. In all evaluation results, latency is measured across all
requests, no matter what page they are for.

Experiments run on Amazon EC2 r5n.4xlarge instances, which have 16 vCPUs and
128GB of memory. The server is always given a dedicated host, while
load-generating clients are split across one or more m5n.4xlarge instances
depending on the desired load factor.

The benchmarks are all ``partially open-loop''~\cite{frank-open-loop}:
clients generate load according to a workload-dictated distribution of
interarrival-times, and has a limited number of backend requests outstanding,
queueing additional requests. This ensures that clients maintain the measurement
frequency even during periods of high latency. The test harness measures offered
request throughput and ``sojourn time''~\cite{open-loop-cautionary-tale}, which
is the delay the client experiences from request generation until a response
returns from the backend.

All experiments measure memory use using the resident virtual memory of the
server process (VmRSS). This measurement therefore includes all indexes, runtime
allocations, and other bookkeeping metadata. For Noria, it also includes the
data stored in the base tables.

Since the benchmarks introduce more data as they run, memory use increases over
the course of each run. Experiments are run for a bit over 5 minutes unless
otherwise specified, and memory measurements are taken at the end of the run.

\begin{inprogress}
  Note about stability of experimental results once all experiments are nailed
  down.
\end{inprogress}

\section{Benefits and Costs of View Materialization}
\label{s:eval:why}

The core argument of this thesis is that partial state makes view
materialization feasible. Bundled up in that argument are several intertwined
questions that must be answered before further evaluation of partial state is
interesting:

\begin{enumerate}
    \item Why is view materialization desirable?
    \item Why is view materialization not feasible currently?
    \item Does partial state improve on this situation?
\end{enumerate}

\begin{figure}[h]
  \centering
  \includegraphics{graphs/lobsters-throughput.pdf}
  \caption{Maximum achieved throughput on Lobsters benchmark with and without
  view materialization. Without view materialization, MySQL must compute query
  results each time. Traditional (full) view materialization runs out of memory
  at $\approx$4.6k pages/second. Partial state allows Noria to reduce memory use
  significantly so that it can achieve higher throughput.}
  \label{f:lobsters-throughput}
\end{figure}

Figure~\ref{f:lobsters-throughput} attempts to explain why view materialization
is desirable. It compares the highest sustainable request load of three
different systems: MySQL, Noria without partial state, and Noria with partial
state. MySQL is run entirely in RAM by running it on a ramdisk, and on its
lowest isolation level. The figure shows the highest Lobsters throughput each
system achieves before its mean latency exceeds 50ms.

View materialization alone (as provided by Noria) improves performance by almost
$12\times$ compared to MySQL, as query results are now frequently cached.
However, without partial state, this performance increase comes at a significant
memory cost. Beyond 4.6k pages/second, it runs out of memory, and cannot
support the workload. With partial state, Noria uses much less memory at a given
load factor, which allows it to support 67\% higher throughput, almost
20$\times$ that of MySQL%
\footnote{The Noria benchmarks are memory constrained, not CPU constrained.
MySQL fully loads all 16 cores at 391 pages per second.}.

\begin{figure}[h]
  \centering
  \includegraphics{graphs/lobsters-memory.pdf}
  \caption{Memory use two minutes into the Lobsters benchmark at 4.6k pages per
  second. Striped bars store base tables on disk using RocksDB.}
  \label{f:lobsters-memory}
\end{figure}

Figure~\ref{f:lobsters-memory} shows the memory use at 4.6k pages per second
with and without partial state. It demonstrates both the issues with full
materialization, and the improvements brought about by partial state. With full
materialization, Noria must store every result for every query in memory. In
contrast, with partial state, Noria stores only frequently accessed results,
which cuts memory use in half.

The memory use reductions with partial state are a direct result of the skew in
Lobsters data popularity and access patterns. Many pages are simply never
visited over the course of the benchmark, and so need not be brought into the
cache. With partial state, Noria also evicts infrequently accessed results,
which further reduces memory use, and ensures that the cache does not eventually
grow to contain all results.

\begin{figure}[h]
  \centering
  \includegraphics{graphs/lobsters-opmem.pdf}
  \caption{Estimated operator state data size two minutes into the Lobsters
  benchmark at 4.6k pages per second. The value indicated includes only the sum
  total size of rows in each operator's state, not data structure overheads,
  indices over the data, or other memory allocations. Base tables are not
  included.}
  \label{f:lobsters-opmem}
\end{figure}

Much of Noria's memory use goes to storing the base tables in memory. Since
partial state cannot evict base table state, this limits how much memory can
potentially be saved. The figure therefore also includes memory use when running
Noria with its durable RocksDB storage backend for base tables. In that
configuration, base tables are kept on disk, not in memory, which makes the
memory savings from partial state more apparent\,---\,the memory use is now
about a third that without partial state.

Various other runtime overheads that partial state cannot eliminate remain, such
as data structure overheads, and allocations for in-flight requests and pending
responses. With diligent memory optimization, this overhead could likely be
further reduced to increase the relative benefits from partial state. To provide
some insight into how far memory use can be reduced,
Figure~\ref{f:lobsters-opmem} shows the total size of the data contained in all
non-base operator state in Lobsters. This metric measures \emph{only} the sum of
the data in each row, and excludes all other memory overheads, such as hash
tables, additional indices, or allocations elsewhere in the application. The
results indicate that partial state in isolation requires only 1.5\% of the
total operator state to be materialized; significantly less than the
$\sfrac{1}{3}$ seen in Figure~\ref{f:lobsters-memory}. This suggests that there
is further potential for reducing partial state's memory footprint.

Since partial state uses less memory, applications that do not need higher
throughput can instead reduce cost by using hosts with less memory. For example,
on AWS EC2, going from a 64GB instance type with 16 cores (\texttt{m5n.4xlarge})
to a 128GB instance type with 16 cores (\texttt{r5n.4xlarge}) comes at a 25\%
price increase. Instances with 256GB of memory only come with 32 cores
(\texttt{r5n.8xlarge}) or more, and come out at twice the price of 128GB.

\section{Rolling Your Own}
\label{s:eval:alts}

Many applications already require lower latency and higher throughput than
straightforward SQL queries against traditional relational databases provide.
Developers often implement manual optimizations to improve their application
performance, and introduce additional complexity into their applications in the
process. These optimizations usually come in one of two forms: denormalization
and caching. This section discusses each of these optimization techniques in
turn, as well as how Noria makes them unnecessary.

\subsection{Denormalization}

The relational database model~\cite{relational} encourages developers to use
``normalized'' schema in which redundant data%
\footnote{The relational model differentiates between strongly and weakly
redundant data\,---\,``redundant'' here refers to strong redundancy.}
that can be derived from other data is not stored. Instead, the model suggests
that derived or computed data be produced on-demand using relational operators
like joins and aggregations. The original paper then adds:

\begin{quote}
Only in an environment with a heavy load of queries relative to other kinds
of interaction with the data bank would strong redundancy be justified in the
stored set of relations.
\end{quote}

As discussed in the motivation section for this thesis, many web applications
fall into exactly this category. Queries are far more common than inserts or
updates, and with a normalized schema they must constantly expend resources to
re-compute such derived data. For this reason, web developers often explicitly
denormalize their schema to include data that would be prohibitively expensive
to compute on-demand.

For example, in Lobsters, each story has a ``hotness'': a score of how popular a
story is, and thus how far up it should appear in listings. This value depends
on a lot of parameters, such as the number of votes, the number of comments,
etc. It would be prohibitively expensive to compute a story's hotness directly
in the queries, especially in the context of computing the front page view,
which requires the hotness for \emph{all} stories to rank them. Instead, the
Lobsters developers add a computed column, \texttt{hotness}, to the
\texttt{story} table. This column is then updated whenever relevant data
changes, such as when:

\begin{itemize}
    \item a story is upvoted or downvoted.
    \item a comment is added to a story.
    \item a comment on a story is upvoted or downvoted.
    \item a comment or vote is deleted.
    \item one story is merged into another.
\end{itemize}

There are several such computed columns in Lobsters. For each one, developers
must inspect all write paths, and change them to ensure that they correctly
update all related computed values. This process is manual and error-prone, but
also necessary: without these manual optimizations, Lobsters running against
MySQL cannot keep up with even a single request per second.

With Noria, such manual denormalization is unnecessary. View materialization
automatically stores and maintains derived data so that it is efficient to
query. The developer can continue to use normalized schemas and queries, and
does not need to modify their application code to manage denormalized columns
and tables.

\subsection{Caching}

If denormalization does not sufficiently improve the application's performance,
the next step is usually to add a cache in front of the database. This cache
often takes the form of a key-value store, like Redis or Memcache, which holds
frequently accessed, computed results. When the application issues a query, it
checks the (fast) cache first, and only if the results are not available in
cache is the (slower) backend consulted.

A dedicated cache speeds up reads that hit, but introduces significant
application complexity. Just like for manual denormalization, all parts of the
application that modify data related to any given cache entry must know to also
invalidate or update the cache. In addition, the developers must ensure
that if multiple clients miss on a given entry, they do not hit the backend
database all at once. This is especially important if a popular entry is
invalidated, as it may cause a ``thundering herd'' effect where a large number
of clients swarm the backend and overwhelm it. Furthermore, since the clients
must now access two separate systems, mechanisms must be in place to ensure that
the cache remains consistent with the underlying data. This is difficult since
data may be updated at any time, including just after a client has fetched the
(then) latest data from the database to update the cache.

Because of the challenges above, implementing caching ``correctly'' requires
highly sophisticated machinery~\cite{facebook-memcache, transactional-cache,
orm-cache, sql-cache}, which developers may not even think to employ. A survey
from 2016 found that 0.3-3.0\% of application code spread across 2.1-10.8\% of
the application's source files is caching-related, and that cache-related issues
make up 1-5\% of all issues~\cite{caching-is-hard}.

With Noria, there is no need to maintain such a query result cache, as Noria's
in-memory materialized views provide high-throughput, low-latency queries
directly from the database. Thanks to partial state, Noria's materialized views
are usable even for applications whose full cache state exceeds the amount of
memory available on the server host. Since Noria automatically maintains the
materialized views, the application also does not need code to manage cache
invalidation or to address challenges like thundering herds.

\section{The Memory/Latency Trade-off}
\label{s:eval:cost}

Partial state's main drawback compared to complete materialized views is
that the results for an application's query may not be known. Or, stated
differently, some reads may miss. When this happens, the system must upquery the
missing state, which takes time and consumes resources otherwise dedicated to
writes. This shows up as increased tail latency for the application: queries
whose results are not known must wait to be computed. The hope with partial
state is that, once the commonly-accessed query results are cached, latency
quickly drops such that going forward only infrequently accessed query results
must be computed on-demand.

\subsection{Warming the Cache}

\begin{figure}[t]
  \centering
  \includegraphics{graphs/lobsters-timeline.pdf}
  \caption{Lobsters latency timeline at 1.5k pages per second across all pages
  with eviction. Figure shows the latency profile seen by the client over time,
  starting at the point when the first query is issued. Time increases along the
  x-axis, with each bin sampling twice as long as the previous one. The measured
  latency for each time bin is plotted on a logarithmic scale on the y-axis.
  Progressively lighter colors include more of the tail.}
  \label{f:lobsters-timeline}
\end{figure}

The cost of these misses is particularly visible when Noria starts with empty
state. This is equivalent to starting a more traditional caching system with an
empty (cold) cache, and having to ``warm'' it by filling in the most popular
entries. To measure this warming period, Figure~\ref{f:lobsters-timeline} shows
the latency profile seen by the Lobsters benchmark described in
\S\ref{s:eval:setup} over time, starting at the point when the first query is
issued. Time increases along the x-axis, and the measured latency for each time
bin is plotted on a logarithmic scale on the y-axis. Progressively lighter
colors indicate values further into the tail.

The figure shows that latency is initially high, but after a few seconds, the
mean and 95th percentile latency drop below 10 milliseconds. By the time a
minute has passed, the 99th percentile has followed suit. Since only a small
portion of the total computed state is cached (as shown in
Figure~\ref{f:lobsters-memory}), this supports the hypothesis that partial state
achieves low latency once the most commonly accessed results are cached.
The latency is primarily determined by the number of queries each page issues,
as each one requires a round-trip to Noria.

\subsection{Impact on Tail Latency}

\begin{figure}[h]
  \centering
  \includegraphics{graphs/lobsters-memlimit-cdf.pdf}
  \caption{CDF of sojourn latency across all Lobsters pages at 1.5k pages per
  second with decreasing permitted memory use. The figure depicts steady-state
  operation\,---\,the benchmark has been allowed to run for two minutes before
  the latency is measured.}
  \label{f:lobsters-mem-latency}
\end{figure}

Partial's trade-off is that of memory use versus tail latency; as you allocate
less memory to Noria, less of your tail can be pre-computed, and thus more of
your requests will be slow as it must be computed on-demand.
Figure~\ref{f:lobsters-mem-latency} shows this trade-off in the steady state%
\footnote{The benchmark runs for two minutes before latencies are sampled.}
of the application. It plots the CDF of the sojourn latency across all requests
with increasingly aggressive eviction. As Noria is asked to
reduce memory use by evicting more aggressively (darker purple lines), more
requests take longer, and the tail grows. In other words, the further you want
to reduce memory use, the more you pay in latency. Without partial
materialization, you do not have this choice\,---\,requests in the tail will
always be fast, but only as long as all the materialized views fit in memory.

\begin{figure}[h]
  \centering
  \includegraphics{graphs/lobsters-durability-cdf.pdf}
  \caption{CDF of sojourn latency across all Lobsters pages at 1.5k pages per
  second with base tables in memory and on disk. The figure depicts steady-state
  operation\,---\,the benchmark has been allowed to run for two minutes before
  the latency is measured.}
  \label{f:lobsters-dur-latency}
\end{figure}

If base tables are not kept in memory, the cost of recomputing missing state
from the data in those base tables increases. Exactly how much depends on the
performance characteristics of the durability backend in use.
Figure~\ref{f:lobsters-dur-latency} shows a CDF of page latencies when Noria's
RocksDB backend is used, with writes going to a ramdisk. Latencies increase by
anywhere from 20-50\%, depending on the number of misses a given page request
experiences.

The reason why the whole curve shifts, rather than just the tail, is that these
CDFs are across \emph{all} the different page types in Lobsters. Each one issues
a different set of queries, and so their total time differs, as does the effect
of a longer tail. The exact shape of this curve, and how it shifts in response
to varying resources, depends on the application in question.

% \begin{inprogress}
% Full materialization is slightly faster than partial materialization because
% SOMETHING SOMETHING either join eviction or faster lookups due to no tombstones?
% \end{inprogress}

\subsection{Memory Use and Throughput}

Memory use can only be reduced so far before the system no longer keeps up with
the offered load. If some of the most frequently accessed (``hot'') query
results are not cached, the system will constantly have to re-compute those
results to satisfy reads that come in shortly after that query result is
evicted. This cache churn increases latency and decreases throughput, often
significantly. Essentially, the system will never finish warming the cache, and
latency will remain at the high levels shown early in
Figure~\ref{f:lobsters-timeline}. For Lobsters, this happens around the 18GB
mark. If the eviction is tuned to be more aggressive than that, Noria can no
longer sustain 1.5k pages per second.

Generally speaking, as throughput increases, so must the memory budget. The
intuition behind this is that the memory budget effectively dictates your hit
rate. The more requests are issued per second, the more misses (in absolute
terms) result from a given hit rate. If those misses in the tail are
distinct, Noria must satisfy \textbf{more} upqueries as load increases, while
also handling that added load.

\begin{listing}[h]
  \begin{minted}{sql}
    SELECT articles.*, COUNT(votes.user_id)
    FROM articles
    LEFT JOIN votes ON (articles.id = votes.article_id)
    GROUP BY articles.id
    WHERE articles.id = ?
  \end{minted}
  \caption{Simplified query for vote counting in Lobsters.}
  \label{l:votes}
\end{listing}

While this correlation between throughput and memory use exists in Lobsters, it
is difficult to show clearly as each page issues many different queries, and
overall load is relatively low. For this reason, the next set of benchmarks use
a simplified version of one particular query from Lobsters shown in
Listing~\ref{l:votes}. It counts the number of votes for an article, and
presents that alongside the article information. The benchmark issues requests
distributed as 99\% reads and 1\% writes (inserts into \texttt{votes}). The
access pattern is skewed such that 90\% of requests access 1\% of keys across
10M articles%
\footnote{The benchmark samples keys from a Zipfian distribution with a skew
factor ($\alpha$) of 1.15}. Load is generated by four clients, and each one
batches requests for a maximum of 10ms to reduce serialization overheads.

\begin{figure}[h]
  \centering
  \includegraphics{graphs/vote-throughput-memlimit.pdf}
  \caption{Achieved throughput vs 95th percentile request latency in vote with
  increasingly aggressive eviction. Offered load increases along the points on
  each line. A near-vertical line indicates that the system no longer keeps up
  with offered load.}
  \label{f:vote-throughput-memlimit}
\end{figure}

Figure~\ref{f:vote-throughput-memlimit} demonstrates the connection between
throughput and memory use. It shows throughput-latency lines for the vote
benchmark with progressively more aggressive eviction. Each point along each
line is a higher offered load; its x-coordinate is the achieved throughput, and
its y-coordinate is the measured median latency. When Noria no longer keeps up,
you see a ``hockey stick'' effect, where achieved throughput no longer
increases, while latency spikes.

The figure shows that as the offered load increases, Noria needs to use more
memory to keep up. If we do not bound memory use, Noria can keep up with higher
load at the cost of caching most of the dataset (about two thirds).

\section{Absolute Cache Performance}
\label{s:eval:kvperf}

Despite how error-prone the approach is, ad-hoc, per-application caching is
still common in practice. And while Noria eliminates most of the developer
burden of getting caching right, it must offer competitive performance with
manually constructed caching solutions to present a viable alternative.

Unfortunately, this is difficult to evaluate, since high-performance solutions
are often developed specifically for a given application, and not available as
general-purpose tools. And effectively applying the general-purpose tools that
\textbf{are} available, like memcache and Redis, requires significant effort on
the part of the application authors (or the evaluators). To manually add caching
support to Lobsters' ~80 queries, including thundering herd mitigation and
incremental updates, would be a significant undertaking.

This fact alone is, in essence, an argument for the Noria approach. The manual
effort involved in making Lobsters use Noria is minimal\,---\,just switch the
code to query Noria instead of MySQL, and you get automatic caching. In many
cases the application code can even be simplified, such as by removing manual
materialization decisions like the story ``hotness'' column described in
\S\ref{s:eval:alts}.

Nevertheless, an experiment to evaluate Noria's absolute performance compared to
a ``real'' cache is necessary. Without such a comparison, Noria can only claim
to be ``faster than MySQL'', but not ``as fast as a cache''.

The next experiment runs the vote benchmark from Listing~\ref{l:votes} against
Redis~\cite{redis}, a popular high-performance key-value store that is commonly
used as a caching backend. In an attempt to approximate how a carefully planned
and optimized application caching deployment might perform, it makes the
following modifications to the benchmark:

\begin{itemize}
 \item Every access hits in cache, to emulate perfect thundering herd mitigation
   and invalidation-avoidance schemes.
 \item Nearly all accesses (99.99\%) are reads, since writes would be
   bottlenecked by the backing store.
 \item Data is not stored anywhere except Redis.
 \item Accesses are batched to reduce serialization cost and increase
   throughput. Specifically, reads are \texttt{MGET}s, and writes are pipelined
    \texttt{INCRBY}s.
\end{itemize}

This is not a realistic use of Redis as a cache, and ignores the complexities of
integrating the cache with the application. It also assumes that cached query
results are never spread across more than one key in the cache. However, it
enables an evaluation that assumes the best about the underlying caching
strategy and system.

\begin{figure}[h]
  \centering
  \includegraphics{graphs/vote-redis.pdf}
  \caption{Achieved throughput vs 95th \%-ile request latency in cache-optimized
  vote. Offered load increases along the points on each line. The vertical
  line indicates $16\times$ the highest Redis throughput, since Redis is
  single-threaded.}
  \label{f:vote-redis}
\end{figure}

Figure~\ref{f:vote-redis} shows a throughput-latency plot that explores the
performance profiles of Redis and Noria under these experimental conditions%
\footnote{Noria runs with the same modified access patterns as outlined for
Redis.}.
% For comparison, it also includes a MySQL + Redis implementation that
% stores votes and articles in MySQL, and uses the na\"ive write-back cache
% strategy outlined above.
Redis is not multi-threaded, and can only use one of the server's 16 cores, so
the figure also includes the Redis performance extrapolated to 16 cores. This is
an over-estimate, since to achieve this performance in practice, the
application's already-perfect caching scheme would need to also shard perfectly.
Noria implements the necessary synchronization internally to take advantage of
all the cores.

The results show that Noria achieves $\approx$42\% of the theoretical 16-core
performance of Redis. Given the idealized nature of this experiment, the exact
absolute numbers should be taken with several grains of salt, but they do
provide an upper bound of sorts for Redis' performance. That Noria approaches
this performance is a good indicator that Noria's cache hit performance is
comparable to that of an ad-hoc caching implementation. And again, Noria does so
while providing rich SQL queries, and without requiring application-specific
caching logic.

\section{Bringing Up New Views}
\label{s:eval:mig}

When the application issues a query that Noria has never seen before, Noria must
instantiate the dataflow for that query, along with any materializations it
might need. Without partial state, the system must do all the work to compute
the full state for the new view, and any internal operator state it depends on,
up front and all at once. And during that time, Noria's dataflow must spend
cycles on computing that new state, slowing down the processing of other
concurrent writes. The new view also cannot serve any reads until all the state
is computed.

\begin{listing}[h]
  \begin{minted}{sql}
    CREATE VIEW scores AS
      -- same vote count (each vote is a 1 rating)
      SELECT votes.article_id, COUNT(votes.user_id) AS score
      FROM votes
      GROUP BY votes.article_id
    UNION
      -- compute total rating score
      SELECT ratings.article_id, SUM(ratings.rating) AS score
      FROM ratings
      GROUP BY ratings.article_id;

    -- new view aggregates across votes and ratings
    SELECT articles.*, SUM(scores.score)
    FROM articles
    LEFT JOIN scores ON (articles.id = scores.article_id)
    GROUP BY articles.id
    WHERE articles.id = ?;
  \end{minted}
  \caption{Updated query for ``rating'' counting in Lobsters.}
  \label{l:ratings}
\end{listing}

Partial state enables such migrations to be instantaneous in many cases\,---\,if
the new view can be made partial it is instantiated as empty, and immediately
made available. It is then filled on demand as the application submits reads.
To demonstrate the different behavior of full and partial materialization for
migrations, the next benchmark makes a modification to the ``vote'' benchmark
from Listing~\ref{l:votes}. It introduces a new table, \texttt{ratings}, which
has ``ratings'' on a scale from 0 to 1 for each article instead of just a binary
0 or 1. It also add a new view, shown in Listing~\ref{l:ratings}, which combines
the existing votes with the new ratings to compute a total article score%
\footnote{By writing the query this way, votes and ratings can co-exist.}.

The benchmark inserts votes and issues the original vote query for 90 seconds,
and then introduces the new table and query (denoted as time 0). From then on,
it issues both votes and ratings, and queries both views without blocking every
10 milliseconds.

\begin{figure}[h]
  \centering
  \includegraphics{graphs/vote-migration.pdf}
  \caption{Time to set up and access a new view. Access pattern is skewed (Zipf;
  $\alpha = 1.15$) across 10M articles. Benchmark runs for 90s prior to
  migration (solid red line). The dashed vertical line denotes the end of the
  migration without partial state.}
  \label{f:vote-migration}
\end{figure}

Figure~\ref{f:vote-migration} plots the cache hit rate for reads from the new
view over time, as well as the write throughput over the course of the
experiment. Without partial state, the new view is not accessible until its
construction finishes after $\approx$23 seconds. During that time, the
application write performance drops substantially, as Noria must compute the
content of the new view.

With partial state, the view is immediately accessible, though its cache hit
rate is initially low. However, since there are a few very popular keys, the hit
rate quickly climbs to over 90\%. Since only results for requested keys are
computed, write throughput is mostly unaffected by the migration%
\footnote{Overall write throughput is lower after the migration since Noria must
now maintain two views, not just one.}.

The figure also exposes another interesting effect of using partial
materialization: increased write throughput. Without partial state, every write
must be processed to completion, since all results are cached. With partial
state, writes for keys that have not been read can be discarded early, as there
is no state in memory that must be updated, which increases throughput.

\section{Skew}
\label{s:eval:patterns}

Partial state is mainly useful if accesses are skewed towards a particular
subset of queries and data. When this is the case, caching a small number of the
application's computed state speeds up a significant fraction of requests. If
this is not the case, the likelihood of missing in the cache is inversely
proportional to the size of the cache, and you would need to cache computations
over most of the data to maintain a decent cache hit rate.

This kind of significant skew is present in Lobsters, which is what allows it to
run smoothly even when only a small fraction of results are cached. Significant
skew shows up across a wide range of other real-world
datasets~\cite{power1,power2,network-skew,large-skew-analysis}, including many
social networks~\cite{network-skew2, community-skew}. In a large public Amazon
dataset~\cite{amazon-skew}, the 100,000 most popular book titles (less than 5\%)
account for roughly 50\% of all book sales, and 75\% of the sales are for the
top 500,000 titles~\cite{zhang2020permutation}.

In the vote benchmark, the decision of which articles to fetch and vote for is
artificially skewed using a Zipfian probability distribution~\cite{zipf}; a
probability model that commonly describes skewed frequency distributions in
natural and random datasets. Specifically, given some number of elements $N$ and
a skew parameter $\alpha$, the normalized frequency of the $k$th element in a
Zipf distribution is given by

\begin{displaymath}
  f\left(k;\alpha,N\right)={\frac {1/k^{\alpha}}{\sum \limits _{n=1}^{N}(1/n^{\alpha})}}
\end{displaymath}

Every time the vote benchmark performs a read or a write, it samples a value,
$k$, in such a way that the likelihood of choosing a given $k$ is given by
$f\left(k;\alpha,N\right)$. A higher value for $\alpha$ means that smaller $k$
values will be sampled more frequently than larger $k$ values, increasing the
skew.

It is difficult to estimate the degree of skew for a complex application ahead
of time. But, because many datasets exhibit skew following something akin to a
Zipfian distribution, an analysis of vote may still yield some helpful
heuristics for application developers.

In vote, after $S$ samples (given by throughput $\times$ period), the expected
number of keys hit is the sum of the probability that each element $k$ is
sampled at least once during the benchmark, given by

\begin{displaymath}
  F(\alpha,N)={\sum \limits _{k=1}^{N} \left(1 - \left(1 - f(k; \alpha, N)\right)^{S}\right)}
\end{displaymath}

\begin{figure}[h]
  \centering
  \includegraphics{graphs/vote-formula.pdf}
  \caption{Probabilistic model of the fraction of 10M keys that are accessed
  (``hot'') during each eviction period ($\approx$ 1s) as throughput increases.
  Each line shows a different amount of skew. Skew (X/Y) denotes that X\% of
  requests come from Y\% of keys. More keys pushes the curve down.}
  \label{f:vote-formula}
\end{figure}

Figure~\ref{f:vote-formula} plots $F(\alpha, N)/N$ after one eviction cycle
($\approx$ 1s) for different degrees of skew ($\alpha$) with $N=10\text{M}$ as
throughput varies. That value corresponds to the expected fraction of keys
accessed between any two eviction cycles, and effectively sets a lower bound on
the fraction of the query results that must be cached. It thus also dictates
minimum memory use. While Noria \emph{could} maintain a smaller fraction of the
query results, the application would likely need those keys again shortly after
evicting them, and Noria would continuously compute and then discard frequently
accessed query results.

Also indicated on the figure is the lowest ratio achieved between operator state
size%
\footnote{This is measured using internal estimates of operator state size, not
VmRSS, so as to only measure the fraction of keys cached.}
with and without partial state in the vote benchmark at 250k operations per
second. The measured value of 0.8\%, or 80k out of 10M articles, gets close to
the expected value of 0.46\%, or 46k out of 10M articles, computed by the formula for the
90/1 skew that the vote benchmark uses. If the eviction is tuned to be even more
aggressive, the benchmark can no longer keep up. This suggests that Noria is
indeed able to function at close to the predicted cache ratio, and that the
model bears some association to observed behavior.

At very high offered load, Noria can rarely get quite as low as the graph
indicates. For example, at 1M operations per second, Noria must maintain 21\% of
keys even though the model predicts that 1.4\% should be sufficient. There are
multiple related reasons for this.

First, Noria currently implements only \emph{randomized} eviction, so frequently
accessed keys will occasionally be evicted. When they do, a large number of
requests must wait for its result to be recomputed. With a less naive eviction
scheme, such as LRU, such evictions can be avoided, and hot keys will never
miss.

Second, more upqueries must be serviced per second. Since upqueries are
performed by the data flow, which is single-threaded along any given path, the
upquery processing itself becomes a bottleneck. To maintain acceptable latency,
Noria is forced to keep many more keys in cache than the model predicts so that
not too many upqueries occur.

And third, Noria's eviction runs at a fixed interval of one second. As offered
load increases, so too does the number of keys read, and the number of keys
cached in that one second. The eviction logic thus has more keys it needs to
evict each time it runs. This in turn takes up more data flow cycles on the
write path that could otherwise be dedicated to serving upqueries.


\chapter{Discussion}

This thesis presents the partially stateful model, as well as its implementation
in Noria. And while the model is complete in isolation, there are a number of
secondary considerations, features, and alternatives that are worth discussing.
Those are discussed in this chapter.

\begin{inprogress}
Some of this should maybe be future work?
\end{inprogress}

\begin{inprogress}
Should some of these go under "Noria background"?
\end{inprogress}

\section{When is Noria not the Answer?}

Noria aims to improve the efficiency of certain classes of database-backed
applications, but is not a one-size-fits-all solution. Noria's materialized
views generally, and partial state specifically, are built specifically for
applications that:

\begin{enumerate}
  \item Are \textbf{read-heavy}. Noria's design is centered around making reads
    cheap, often at the cost of write performance. For workloads where writes
    are as frequent, or more frequent, than reads, other systems will work
    better.
  \item Tolerate \textbf{eventual consistency}, at least for large parts of the
    application's workload. Much of Noria's performance advantages over other
    systems stems from the relaxed consistency model. If much of the
    application's workload requires stronger consistency guarantees, there is
    little for Noria to speed up.
  \item Experience \textbf{good locality}. If the application's access patterns
    are all completely uniform, caching is much less impactful unless \emph{all}
    results are cached. In that case, partial state, and the complexity it
    introduces, provides little value. Instead, Noria works best if data and
    access distributions are skewed, and demonstrate good temporal and spatial
    locality.
  \item Have \textbf{non-trivial computed state}, both in size and complexity.
    If all computed state fits in a small amount of memory, a materialized view
    system without partial state would work just as well. If all queries are
    simple point queries without aggregations or joins, Noria's incremental
    cache update logic is unnecessary, and a simpler cache invalidation scheme
    may work better.
\end{enumerate}

It is also worth noting that Noria may not perform as well as a fully developed,
manually tuned backend caching system. While Noria would allow the removal of
caching logic from the application, its general-purpose architecture may miss
out on application-specific optimizations implemented by a tailor-built system.

\section{Efficient Migrations}

Section~\ref{s:eval:mig} demonstrated that partial state can make certain
migrations efficient. This requires that the view \emph{can} be partial, as
per the discussion above. But even for views that can be partial, work may be
required in order to make upqueries for that view efficient. This work generally
means adding an index to some existing state, which requires scanning the data
stored in that view. Constructing an index tends to be significantly faster than
computing the full cached results of the new view, but it is a non-trivial cost
nonetheless.

For example, consider the query in Listing~\vref{l:karma} when added to the vote
benchmark from Listing~\vref{l:votes}. To simplify the argument, assume that the
\texttt{VoteCount} view is for some reason \emph{not} partially stateful (i.e.,
it holds all the rows). For upqueries of the new view to be efficient, it must
be possible to query all the articles (along with their vote counts) for a given
author in the \texttt{ViewCount} view that existed previously. This means we
must add an index on the \texttt{author} column of that view's state, which is
costly.

\begin{listing}[h]
  \begin{minted}{sql}
    SELECT VoteCount.author, SUM(VoteCount.nvotes) AS karma
    FROM VoteCount -- the view from the vote benchmark
    GROUP BY VoteCount.author
    WHERE VoteCount.author = ?
  \end{minted}
  \caption{Query that computes the sum total score of a user's articles
  (their ``karma'').}
  \label{l:karma}
\end{listing}

\begin{inprogress}
  If \texttt{VoteCount} is partial, it \emph{is} free, because indexes for
  partially stateful views always start out empty. Each index of a partially
  stateful view is completely independent! There are no "shared" rows across
  indexes. But we probably want there to be.
\end{inprogress}

A comparison with what would happen when using a traditional relational database
is useful here. When the application developer decides that they want to run
this new query, they have two choices: either compute it on-demand, or
denormalize the schema by adding a new computed ``karma'' column to the
(hypothesized) \texttt{users} table. Neither option is great. The former is slow
to execute, and the latter requires computing the karma for every article.
The index Noria must construct for efficient upqueries is cheaper to construct
than such a computed \texttt{karma} column, which makes Noria's scan seems
reasonable.

% SELECT COUNT(*) FROM table;

Whether Noria \emph{always} does no more work than what a developer would
make a traditional relational database do if they wanted to make a view
efficient to query remains an open question.

\section{Ordered State}
\label{s:disc:ordered}

Certain ordered operations, like max aggregations (\texttt{SELECT MAX}) and
top-k-style queries (\texttt{ORDER BY LIMIT}), occasionally require re-fetching
underlying state as the data changes. If the maximum value in a max aggregation
or a row in a top-k view is removed, the new view content can only be determined
by re-evaluating the query.

The necessary upquery can be performed efficiently if the underlying state is
maintained in the appropriate order, but Noria does not currently support the
necessary ordered indexes. Instead, Noria provides approximate versions of such
operators. In particular, Noria's top-k operator maintains the top $2k$ items,
so that if an item is removed, the top $k$ items are still known. To get back to
$2k$ (to allow future removals), the operator fills the top view with the
highest rows it has seen so far.

This scheme avoids the need for upqueries, and works well as long as removals
from the top list are uncommon and the top list rotates over time. Otherwise,
the approach is brittle; if many top rows are removed, or if the top is changing
very infrequently, the top list may eventually hold none of the actual top
items. Support for ordered indexes, and limited upqueries against those indexes,
would fix this problem.

\section{Ranged Upqueries}
\label{s:disc:ranged}

Throughout this thesis, upqueries have been described in terms of point lookups
of the form \texttt{WHERE x = ? AND y = ?}. However, the design of partial state
is also amenable to supporting ranged queries (\texttt{WHERE x = ? AND y < ?}).
Much of the necessary work lies in changing the appropriate index structures and
including range information in upqueries, which is all straightforward. The
trickiest part of the change is to ensure that future updates are not dropped if
they fall within a requested range. For example, consider the following course
of events:

\begin{enumerate}
  \item An insert arrives with \texttt{x = 42}.
  \item An upquery arrives with \texttt{x < 50}.
  \item An insert arrives with \texttt{x = 49}.
\end{enumerate}

The second insert must be forwarded downstream so that it will update the
materialized state for \texttt{x < 50}. For Noria to realize that this is the
case, it must ``remember'' the \texttt{x < 50} upquery. More generally, it must
remember what \emph{ranges} of values are present downstream, not just what
individual keys. The solution here is to use an interval tree to track which
parts of the key space is present. An interval tree efficiently stores, merges,
and splits ranges as new ones are introduce (by new upqueries) and retired (by
evictions).

\section{Sharding Upquery Explosions}

An unfortunate phenomenon manifests itself for certain queries when partial
state and sharding combine in detrimental ways. If R is sharded differently from
Q; Q is partial; and Q's materialized ancestor, P, is sharded differently from
Q, then a miss in R may cause $k^2$ upqueries to be issued to P, where $k$ is
the sharding factor. The miss in R generates an upquery to every shard of Q, and
every shard of Q sends an upquery to every shard of P.

The three modifications above are sufficient to ensure that this situation is
handled \emph{correctly}, but additional research is needed to reduce the number
of upqueries needed. One promising avenue may be to optimize for the case where
all the shards of Q miss. If every shard of Q knows that every other shard will
upquery P, they may be able to coordinate the upqueries such that any given key
is only upqueried for once. The sharder node can then ensure that the upquery
results are sent to all the shards. This is left for future work.

\section{Emulating Partial State}

A natural question to ask is whether the benefits of partial state can be
achieved without the complexity that upqueries introduce. In particular, can a
dataflow system that supports only full materialization emulate partial state
effectively? Thoroughly exploring the answers to this question may be worth a
thesis in its own right, but some of the more obvious approaches are discussed
below.

\subsection{Lateral Joins}

The commercial materialized view stream processor Materialize~\cite{materialize}
supports \emph{lateral} joins~\cite{lateral-join}, which is described as

\begin{quote}
  [A] join modifier [that] allows relations used in a join to ``see'' the
  bindings in relations earlier in the join.
\end{quote}

In particular, lateral joins let the application author write a query that has
access to the contents of some unrelated table. For example,
Listing~\vref{l:emulate-partial-vote} shows how a lateral join can be used to
emulate a partially materialize vote count view like the one from
Listing~\vref{l:votes}. The idea here is to have a table of ``filled'' keys, and
have the results only for those keys be included in the final materialized view.

\begin{listing}[h]
  \begin{minted}{sql}
CREATE MATERIALIZED VIEW VoteCount AS
SELECT article_id, votes FROM
  (SELECT DISTINCT article_id FROM queries) filled,
  LATERAL (
      SELECT COUNT(*)
      FROM votes
      WHERE article_id = filled.article_id
  );
  \end{minted}
  \caption{Using a lateral join to emulate partial state in vote.}
  \label{l:emulate-partial-vote}
\end{listing}

This approach works well to emulate partial state in simple situations, but
requires significant manual effort for a large application. In Lobsters, for
example, the application author must re-write their applications to use such
lateral joins, and must include application logic to maintain the auxiliary
tables used to indicate what keys are materialized. It may be possible to
automate this process, though doing so requires research.

Effort notwithstanding, emulating partial state in this way also presents an
``all or nothing'' choice for applications for a given key. Either, all state
for that key is computed, or none of it is. With partial state, the state for a
key in the ultimate materialized view can be evicted without also evicting the
current vote count. The former may be significantly larger than the latter,
since it includes other columns, but is cheap to recompute. The latter on the
other hand is small, but potentially expensive to re-compute.

\subsection{State Sharing}

Partial state allows a single query of the form \texttt{WHERE x = ?} to satisfy
lookups for any value of \texttt{?}. Without partial state, the system has two
options: to remove the filter on \texttt{x} from the query and filter after the
fact, or to instantiate a separate query for each concrete value of \texttt{?}
supplied by the application. The former uses a significant amount of memory, but
is also complicated to get right; \texttt{x} may for example affect what values
are aggregated together. The latter is simpler, and uses less memory, but
requires duplicating the dataflow operators for each query, and keeping separate
state for each one.

Recent work introduced arrangements~\cite{arrangements} as a way to mitigate
this problem. Arrangements allow sharing indexes and state across related
operators to avoid duplication. However, even with arrangements, the system may
execute the same computation over a given input record more than once if it is
needed by more than one instance of a query. Furthermore, eviction is made more
difficult with arrangements, as the system must update the entire arrangement to
accommodate any new or evicted parameter value. Noria supports joint query
optimization~\cite{noria}, which combined with arrangements could reduce much of
the duplicated effort by instantiating each query multiple times, though this
does not improve the eviction process.

\section{Fault Tolerance}

If an operator's state is lost, Noria's current recovery strategy is to remove
and re-introduce the operator, and all of its descendants, as if they were new
queries. This can happen because the Noria worker hosting that operator fails,
or simply because the system is restarted. This scheme works, but means that any
past materialization work is lost and must be re-done.

A mechanism for taking snapshots of materialized state that can be recovered
later would help mitigate this. However, such a design also requires care to
ensure that any state populated \emph{since} the snapshot is correctly
incorporated. In particular, if downstream state now includes entries that
reflect data missing from the snapshot, the system must evict that downstream
state. Otherwise, updates for that data will be discarded at the recovered
operator when it discovers that the related state is missing in its state.

\section{Consistency}

Noria provides weaker consistency guarantees than many existing dataflow and
view materialization systems. This has implications for how applications use
Noria, and what behavior the application may observe.

\subsection{Write Latency == Staleness}

By design, Noria's read and write paths are disconnected from one another: reads
can usually proceed even if the write path is busy. This is both the reason why
Noria's read performance is so high, and why it gives weaker consistency
guarantees that competing systems. For example, on a 32-core machine, the
application may experience a write throughput ceiling at a few hundred thousand
updates per second, as the write path is processed by only a small number of
cores. Meanwhile, reads can happen across any number of cores; even if the write
path is entirely saturated, Noria may be able to handle millions of additional
reads per second.

While a saturated write path does not slow down the execution of queries whose
results are materialized, it does affect the read path in two important ways:
miss-to-hit time and result staleness. If a query misses, the dataflow must
compute and populate the missing state so that the read can proceed. This is the
same dataflow that handles writes, so the time until the missing read hits
instead will increase if the dataflow is busy. Similarly, while queries that do
not miss can proceed immediately, the returned results will not reflect updates
that have not yet been processed by the dataflow. Therefore, if the dataflow is
busy, the time between when an update is issued and when it is reflected in
later queries will increase.

\subsection{Transactions}

Web applications sometimes rely on database transactions, e.g., to atomically
update pre-computed values. Noria does not implement transactions, though its
support for derived views often obviates the need for them. For example, web
applications often use transactions to keep denormalized schemas synchronized: a
``like count'' column in the table that stores posts or an ``average rating''
column in the table that stores products. Noria obviates the need for such
denormalizations, and the transactions needed to maintain them, by automatically
ensuring that computed derived values are kept up to date with respect to the
base data.

\subsection{Stronger Consistency}

Noria is eventually consistent, and so is the partial state implementation
outlined in this thesis. That said, adding partial state to a system with
stronger consistency guarantees should not require extensive changes. In fact,
parts of the design could likely be simplified; union buffering, for example,
would likely no longer be necessary, and could be replaced with some kind of
multi-versioned concurrency control.

\section{Upstream Database Integration}

Existing applications that wish to adopt Noria may not want to adopt it
wholesale. They may wish to continue using their existing data backend because
they rely on its transactional properties for parts of their workload, because
they have existing backup solutions in place, or simply to make the transition
incrementally.

The most straightforward way to add Noria to an existing application backend is
to feed all changes to the primary database tables into Noria. Noria will then
maintain its copies of the base tables, with indexes it manages itself. However,
this has the downside of duplication all of the application's data between the
primary backend and Noria.

A more attractive alternative is to integrate the existing backend into Noria's
base tables. Noria would still have to be notified as changes are made to the
data so that it can propagate those changes to the maintained views, but that
data would not also have to be stored in Noria's base tables.

Unfortunately, implementing this design naively introduces a race condition:
there is now a window of time where a change that has been made to the base data
is visible to upqueries to the base tables, but the corresponding update has not
yet entered the dataflow. This is a problem, because if an upquery response
reflects that new data, and then an update arrives and adds that same data, the
data will be reflected twice!

A possible solution is to take a page out of the multi-version concurrency
control playbook, and ensure that lookups into base table state do not see the
effects of any updates that have not yet passed through its Noria operator
equivalent. Ideally, this would be based on the existing transactional
capabilities of the upstream database, but it may also be possible to emulate
using an audit table that records table changes. This is left for future work.

\section{Maintaining Downstream Systems}

Since Noria internally propagates updates that reflect deltas to past state, a
natural idea is to extend those deltas to downstream systems. For example, Noria
could notify a reactive web application when the result set for the view it is
currently showing is modified, and include in that notification what exactly
changed. In response, the application could reflect that change, all without
sending another query to the database.

Extending Noria in this way raises an interesting question around partial state.
In particular, what happens if an application ``subscribes'' to a query, and
then that query's result set is evicted? Since it is evicted, Noria will not
maintain it any longer, and the application's view will grow stale. Similarly,
what happens if the application attempts to subscribe to a query whose results
are not currently known? Or, what if the application goes offline briefly, and
now wishes to gather only the changes to the result set since it was last
online?

It may be that the solution here is simple\,---\,provide a query-and-subscribe
operator that populates missing state if needed, and then ensure that results
for outstanding subscriptions are never evicted. The view could also retain a
log of recent changes to the view to replay if a slightly stale client wants to
catch up.

\section{Eviction Strategy}

Partial state enables Noria to evict state that is infrequently accessed. It
does not dictate any particular eviction strategy as long as the partial state
invariants are maintained. In particular, if state is evicted at some operator,
any downstream state derived from the evicted state must also be evicted.

This thesis does not attempt to innovate in the space of eviction schemes, and
implements simple randomized eviction: when memory use exceeds a given
threshold, keys are evicted randomly from the three largest states in each of
the three largest domains. The number of keys is chosen proportionally to the
size of each state. This scheme works decently, and requires little coordination
or complexity, but suffers when the system runs close to capacity. Frequently
accessed keys may still be evicted due to pure chance, and when that happens the
system falls behind.

To push Noria to higher load, a smarter eviction strategy like least-recently
used should be implemented. The primary obstacle to overcome is that evictions
must happen in the dataflow write path, but the information needed to inform
eviction decisions usually come from the read path. Care must be taken to avoid
excessive synchronization between these, otherwise Noria's read performance
would be bottlenecked by the performance of the write path.

\section{Column-Based Storage}

Noria's in-memory storage is unoptimized. Specifically, every row in every state
is allocated in its own vector. This stresses the memory allocator, and
introduces non-trivial memory overhead. Since Noria knows the schema of each
view in advance, and all rows in the view have the same schema, a column-based
storage format would likely be a much better fit for many views. Noria could
even use heuristics to choose between row- and column-based storage depending on
the semantics of each operator.

\section{Cursors}

Websites frequently have paginated listings, or pages that are filled in with
more content as the user scrolls. Behind the scenes, these techniques are both
implemented using the same abstract mechanism: the cursor. There are many ways to
implement a cursor, but the most common is using the SQL \texttt{LIMIT}
operator.

On page one of a listing page with 10 results per page, the listing query is run
with \texttt{LIMIT 10}. On page two, the same query is run either with
\texttt{OFFSET 10} to skip the results from page one, or with a \texttt{WHERE}
clause that excludes the results that have already been shown. For example, if
the listing query orders results by id, the \texttt{WHERE} clause could be
\texttt{id > ?} where \texttt{?} is the last id on the previous page.

Some databases support persistent cursors as well. With a persistent cursor, the
database keeps track of what subset of the results for a query the application
has already seen, and the application can ask to read more results from that
existing cursor.

Noria currently cannot represent cursors like these since it does not maintain
the order of in-memory state (\S\ref{s:disc:ordered}). \texttt{OFFSET} might not
skip the same results as shown on the previous page, and \texttt{WHERE x > ?} is
not supported. If support for ordered state was added, Noria would support these
types of queries much like existing databases.

To make paginated queries \emph{partial}, additional challenges must be solved.
First, ranged upqueries must be implemented to support \texttt{x > ?}
conditionals (\S\ref{s:disc:ranged}). Then, a decision must be made as to how
\texttt{LIMIT} should interact with upqueries. There are two primary design
options: \emph{post-limiting} and \emph{pre-limiting}.

In a post-limited design, the query is executed without pagination-related
clauses internally, and all of its results are materialized. The limit and
offset are then applied ``at the end'': when a query execution request comes in,
only an appropriate subset of the materialized results are returned. This
solution requires no changes to the partial state logic, but also makes it
necessary to materialize all pages of each query result, even if only the first
few pages are ever accessed. Realistically, a solution that takes this approach
would therefore also include a hard upper limit on how many results are
materialized. Twitter takes an approach like this, where there is a fixed end to
each timeline that the user cannot scroll past.

In a pre-limited design, materialized state includes only results for pages that
have been accessed. This is attractive since it uses less memory, and fewer
results must be maintained. But, it also requires more complex changes to
partial state. In particular, operators must now have a way to determine if a
state change causes records to appear in, or disappear from, materialized pages
downstream. If they do not, updates may be discarded early even though they
would change downstream materialized state.

Furthermore, since intermediate operators may remove (e.g., filters) or add
(e.g., joins) rows to the resultset, the limit requested by the application may
not map directly to the number of results yielded by the corresponding upquery.
Therefore, page-specific upqueries may need to run multiple ``iterations'' to
fetch additional results if the first response did not return enough rows.

\begin{inprogress}
  Time windowing
\end{inprogress}

\section{Partial Key Subsumption}

Noria's implementation of partial state does not currently take advantage of
situations where upquery keys overlap. For example, consider the case of an
operator X where one downstream operator upqueries on column A, and another
upqueries on the pair of columns A and B. X currently keeps two indices: one on
A, and one on A+B. Each index keeps track of missing entries independently. So,
even if we previously executed and filled in an upquery for A = 3, a subsequent
request for A = 3, B = foo could miss and cause another upquery to be issued.
The operator has sufficient information that it should be able to resolve this
index miss locally, but Noria does not currently implement this optimization.


\chapter{Future Work}

\input{08-future-work.tex}

\appendix
\chapter{Noria In Simpler Terms}

\input{A1-discussion.tex}

\backmatter

% single spacing for bibliography
\begin{spacing}{1}
\printbibliography
\end{spacing}

\end{document}
